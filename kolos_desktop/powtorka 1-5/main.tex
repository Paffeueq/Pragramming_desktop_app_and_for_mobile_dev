\documentclass[12pt, a4paper, utf8]{book}
\usepackage[T1]{fontenc}
\usepackage[polish]{babel}
\usepackage{geometry}
\usepackage{graphicx}
\usepackage{listings}
\usepackage{xcolor}
\usepackage{hyperref}
\usepackage{fancyhdr}
\usepackage{tocloft}

\geometry{margin=1in}

% Konfiguracja listingów
\lstset{
	basicstyle=\ttfamily\small,
	keywordstyle=\color{blue},
	commentstyle=\color{gray},
	stringstyle=\color{red},
	breaklines=true,
	showstringspaces=false,
	tabsize=4,
	frame=single,
	backgroundcolor=\color{lightgray! 20}
}

\title{\textbf{Kompletny Materiał do Nauki} \\ Programowanie . NET}
\author{Przygotowanie do egzaminu}
\date{\today}

\pagestyle{fancy}
\fancyhf{}
\fancyhead[L]{Materiały do nauki . NET}
\fancyhead[R]{\thepage}
\setlength{\headheight}{15pt}

\begin{document}
	
	\maketitle
	\tableofcontents
	\newpage
	
	% Include poszczególne rozdziały
	\chapter{Programowanie AI - Copilot, VS Code}

\section{Wstęp}
GitHub Copilot to asystent AI w Visual Studio Code, który wspomaga pracę programisty.  Zajęcia skupiają się na zaawansowanych funkcjach, od Agent Mode po personalizację poprzez instrukcje. 

\section{Agent Mode i jego zastosowania}

\subsection{Definicja}
Agent Mode to tryb, w którym Copilot działa automatycznie bez konieczności potwierdzania każdego kroku. Różni się od klasycznego trybu interaktywnego. 

\subsection{Praktyczne zastosowania}
\begin{itemize}
	\item \textbf{Build i test}: automatyczne budowanie i testowanie projektów
	\item \textbf{Długotrwałe zadania}: delegowanie złożonych operacji
	\item \textbf{Code review}: analiza kodu przez agenta
	\item \textbf{Generowanie endpointów API}: szybkie tworzenie usług
\end{itemize}

\subsection{Tryby pracy}
\begin{description}
	\item[Interaktywny] Copilot prosi o potwierdzenie każdego kroku
	\item[W pełni automatyczny] Agent wykonuje całe zadanie bez przerw
\end{description}

\section{Allow / Deny List – bezpieczeństwo}

\subsection{Rola listy dozwolonych komend}
Allow List zapewnia bezpieczeństwo przez limitowanie komend, które agent może wykonać automatycznie.

\subsection{Praktyczne przykłady}
\begin{lstlisting}
	// Dozwolone komendy
	- dotnet build
	- dotnet test
	- dotnet run
	
	// Zakazane komendy
	- rm -rf . 
	- del *.*
\end{lstlisting}

\subsection{Konfiguracja}
\begin{enumerate}
	\item Otwórz Settings → Copilot → Experimental
	\item Dodaj komendy do Allow List:
	\begin{itemize}
		\item \texttt{dotnet build}
		\item \texttt{dotnet test}
	\end{itemize}
	\item Po skonfigurowaniu projekt buduje się automatycznie bez pytania
\end{enumerate}

\section{Zarządzanie limitami - Max Requests}

\subsection{Wpływ na wydajność}
Domyślnie Max Requests wynosi 20.  Limit wpływa na:
\begin{itemize}
	\item Ilość automatycznych operacji
	\item Płynność pracy z agentem
	\item Czas wykonywania długotrwałych zadań
\end{itemize}

\subsection{Praktyczne ustawienia}
\begin{table}[h]
	\begin{tabular}{|l|l|l|}
		\hline
		\textbf{Projekt} & \textbf{Max Requests} & \textbf{Uzasadnienie} \\
		\hline
		Mały projekt & 20 & Domyślnie wystarczy \\
		Średni projekt & 50-100 & Lepszy flow pracy \\
		Duży projekt & 100+ & Złożone zadania \\
		\hline
	\end{tabular}
\end{table}

\section{Copilot Instructions - personalizacja}

\subsection{Automatyczne generowanie}
\begin{enumerate}
	\item Otwórz chat Copilota
	\item Kliknij "Customize chat" → "Generate Instructions"
	\item Copilot skanuje projekt i tworzy \texttt{copilot-instructions.md}
\end{enumerate}

\subsection{Zawartość wygenerowanych instrukcji}
\begin{itemize}
	\item Struktura projektu
	\item Komendy budowania i uruchamiania
	\item Konwencje kodu (PascalCase, camelCase)
	\item DTO z atrybutami JSON
	\item Reguły CSS
\end{itemize}

\subsection{Przykład własnych reguł}
\begin{lstlisting}
	# Copilot Instructions
	
	## Konwencje kodu
	- Klasy i metody: PascalCase
	- Zmienne: camelCase
	- Pola private: _camelCase
	
	## DTO
	Wszystkie DTO mają atrybuty JSON:
	[JsonPropertyName("fieldName")]
	
	## C# Best Practices
	- Używaj using statements
	- Dokumentuj publiczne metody XML comments
\end{lstlisting}

\subsection{Rozdzielanie zasad dla backendu i frontendu}
Utwórz oddzielne pliki:
\begin{itemize}
	\item \texttt{. github/csharp-guidelines.md}
	\item \texttt{.github/blazor-guidelines.md}
\end{itemize}

\section{Snooze Completions - balans AI/samodzielne kodowanie}

\subsection{Kiedy wyciszyć Copilota}
\begin{itemize}
	\item Nauka nowych konceptów
	\item Ćwiczenie świadomego programowania
	\item Unikanie uzależnienia od AI
\end{itemize}

\subsection{Jak korzystać}
\begin{enumerate}
	\item Kliknij ikonę Copilota w pasku statusu
	\item Wybierz "Snooze" (np. 5 minut)
	\item Pracuj samodzielnie
	\item Po upłynięciu czasu podpowiedzi wracają
\end{enumerate}

\section{Custom Chat Modes}

\subsection{Beast Mode}
Specjalny tryb, w którym AI rozbija zadanie na kroki i realizuje je sekwencyjnie.

\subsection{Konfiguracja}
\begin{enumerate}
	\item Utwórz \texttt{. github/chat-modes/beast.json}
	\item Uruchom Agent Mode
	\item Wybierz Beast Mode z listy trybów
	\item Poproś o dodanie endpointu API
\end{enumerate}

\section{Coding Agents - wirtualny współprogramista}

\subsection{Przypisywanie zadań}
\begin{itemize}
	\item Utwórz issue na GitHub
	\item Przypisz je do @copilot
	\item Agent automatycznie tworzy branch i PR
\end{itemize}

\subsection{Delegowanie pracy}
\begin{enumerate}
	\item W VS Code otwórz panel agenta
	\item Kliknij +, dodaj nowe zadanie
	\item Agent tworzy nowy branch i sesję
	\item Pracuje w tle, podczas gdy ty kodując lokalnie
\end{enumerate}

\subsection{Code review od agenta}
\begin{itemize}
	\item Dodaj Copilota jako reviewera PR
	\item Agent sugeruje ulepszenia
	\item Może proponować użycie ILogger zamiast Console. WriteLine
\end{itemize}

\section{Integracja z GitHub Actions}

\subsection{Automatyzacja setupu środowiska}
Plik \texttt{.github/workflows/copilot-setup-steps.yml}:
\begin{lstlisting}
	name: Setup Steps
	on: [push]
	jobs:
	setup:
	runs-on: ubuntu-latest
	steps:
	- uses: actions/checkout@v3
	- name: Setup .NET 9
	uses: actions/setup-dotnet@v3
	with:
	dotnet-version: '9.0'
	- name: Restore packages
	run: dotnet restore
\end{lstlisting}

\section{Zarządzanie modelami AI}

\subsection{Dostępne modele}
\begin{itemize}
	\item \textbf{GPT-5}: Najnowszy, najlepsze rezultaty
	\item \textbf{GPT-5 mini}: Szybszy, mniej zasobów
	\item \textbf{Starsze modele}: Można usunąć
\end{itemize}

\subsection{Zmiana modelu}
\begin{enumerate}
	\item VS Code 1.103+
	\item Otwórz model selector
	\item Wybierz preferowany model
	\item Usuń starsze modele z listy
\end{enumerate}

\section{MCP Auto Start i narzędzia developerskie}

\subsection{Rola MCP}
MCP (Model Context Protocol) udostępnia dodatkowe narzędzia dla Copilota. 

\subsection{Konfiguracja}
\begin{enumerate}
	\item Settings → MCP Auto Start → ustaw "always"
	\item Configure Tools → przejrzyj dostępne narzędzia
	\item Wyłącz nieużywane, aby zmniejszyć szum
\end{enumerate}

\section{Statystyki użycia AI}

\subsection{Analiza wydajności}
\begin{itemize}
	\item AI vs typing - co naprawdę daje Copilot?
	\item Eksperymenty porównawcze
	\item Eksport danych do analizy
\end{itemize}

\subsection{Interpretacja danych}
\begin{enumerate}
	\item Włącz AI Stats
	\item Sprawdź wykres w dolnym rogu (24h)
	\item Porównaj produktywność z/bez AI
\end{enumerate}

\section{Pytania kontrolne}

\begin{enumerate}
	\item Jaka jest różnica między trybem interaktywnym a w pełni automatycznym?
	\item Dlaczego Allow List jest ważny dla bezpieczeństwa?
	\item Jak wygenerować automatyczne instrukcje dla projektu?
	\item Kiedy warto używać Snooze Completions?
	\item Jak przypisać zadanie do Coding Agent?
	\item Jakie są dostępne modele AI w Copilot?
\end{enumerate}
	\newpage
	\chapter{Tworzenie API REST}

\section{Wstęp}
REST (Representational State Transfer) to architektura do tworzenia usług sieciowych. API REST używa standardowych czasowników HTTP i statusów do komunikacji między klientem a serwerem.

\section{Stworzenie nowego Web API w Visual Studio}

\subsection{Inicjalizacja projektu}
\begin{lstlisting}
	dotnet new webapi -n ShopAPI. Api
	cd ShopAPI.Api
	code .
	dotnet run
\end{lstlisting}

\subsection{Gdzie uruchomić się aplikacja}
\texttt{https://localhost:7294/swagger/index.html}

\section{HTTPS i certyfikat SSL}

\subsection{Weryfikacja certyfikatu}
\begin{lstlisting}
	dotnet dev-certs https --check
	dotnet dev-certs https --trust
\end{lstlisting}

\subsection{Omijanie zabezpieczenia SSL w Chrome}
\begin{enumerate}
	\item Wpisz: \texttt{chrome://flags/\#allow-insecure-localhost}
	\item Ustaw na "Enabled"
	\item Zrestartuj Chrome
\end{enumerate}

\section{Testowanie API}

\subsection{Narzędzia testowania}
\begin{description}
	\item[Swagger] Interfejs webowy, intuicyjny, bezpośrednio w aplikacji
	\item[curl] Minimalistyczne, działające z konsoli, idealne dla skryptów
	\item[Postman] Rozbudowane, możliwość zapisywania kolekcji
	\item[Pliki . http] Native w VS Code, idealnie dla . NET developerów
	\item[Przeglądarka] Tylko dla GET, szybka weryfikacja
\end{description}

\subsection{Przykład z Swagger}
API automatycznie generuje dokumentację na \texttt{/swagger}

\subsection{Przykład z curl}
\begin{lstlisting}
	# GET
	curl https://localhost:7294/api/cities
	
	# POST
	curl -X POST https://localhost:7294/api/cities \
	-H "Content-Type: application/json" \
	-d '{"name":"Warsaw","country":"Poland"}'
\end{lstlisting}

\section{Czasowniki HTTP i dobre praktyki REST}

\subsection{Podstawowe metody}
\begin{table}[h]
	\begin{tabular}{|l|l|l|}
		\hline
		\textbf{Metoda} & \textbf{Opis} & \textbf{Przykład} \\
		\hline
		GET & Pobieranie danych & GET /api/cities \\
		POST & Dodawanie nowych danych & POST /api/cities \\
		PUT & Aktualizacja całego zasobu & PUT /api/cities/1 \\
		PATCH & Częściowa aktualizacja & PATCH /api/cities/1 \\
		DELETE & Usunięcie zasobu & DELETE /api/cities/1 \\
		\hline
	\end{tabular}
\end{table}

\subsection{Zaawansowane metody}
\begin{description}
	\item[HEAD] Pobiera tylko nagłówki (sprawdzenie, czy zasób istnieje)
	\item[OPTIONS] Zwraca dostępne metody dla endpointu
\end{description}

\subsection{Idempotencja}
\begin{itemize}
	\item \textbf{Idempotentne}: GET, PUT, DELETE (wielokrotne wywołanie = ten sam efekt)
	\item \textbf{Nie idempotentne}: POST (za każdym razem tworzy nowy zasób)
\end{itemize}

\section{Przykład CitiesController}

\subsection{Implementacja endpointów}
\begin{lstlisting}
	[ApiController]
	[Route("api/[controller]")]
	public class CitiesController : ControllerBase
	{
		private static List<CityDto> cities = new();
		
		// GET /api/cities
		[HttpGet]
		public ActionResult<List<CityDto>> GetAll() 
		=> Ok(cities);
		
		// GET /api/cities/{id}
		[HttpGet("{id}")]
		public ActionResult<CityDto> GetById(int id)
		{
			var city = cities.FirstOrDefault(c => c.Id == id);
			if (city == null) return NotFound();
			return Ok(city);
		}
		
		// POST /api/cities
		[HttpPost]
		public ActionResult<CityDto> Create([FromBody] CityDto dto)
		{
			var newCity = new CityDto 
			{ 
				Id = cities.Count + 1, 
				Name = dto.Name, 
				Country = dto. Country 
			};
			cities.Add(newCity);
			return CreatedAtAction(nameof(GetById), 
			new { id = newCity. Id }, newCity);
		}
		
		// PUT /api/cities/{id}
		[HttpPut("{id}")]
		public IActionResult Update(int id, 
		[FromBody] CityDto dto)
		{
			var city = cities.FirstOrDefault(c => c.Id == id);
			if (city == null) return NotFound();
			
			city.Name = dto.Name;
			city.Country = dto.Country;
			return NoContent();
		}
		
		// PATCH /api/cities/{id}
		[HttpPatch("{id}")]
		public IActionResult PartialUpdate(int id, 
		[FromBody] JsonPatchDocument<CityDto> patch)
		{
			var city = cities.FirstOrDefault(c => c.Id == id);
			if (city == null) return NotFound();
			
			var cityDto = new CityDto 
			{ 
				Name = city.Name, 
				Country = city.Country 
			};
			patch.ApplyTo(cityDto);
			
			city.Name = cityDto.Name;
			city.Country = cityDto.Country;
			return NoContent();
		}
		
		// DELETE /api/cities/{id}
		[HttpDelete("{id}")]
		public IActionResult Delete(int id)
		{
			var city = cities.FirstOrDefault(c => c.Id == id);
			if (city == null) return NotFound();
			
			cities.Remove(city);
			return NoContent();
		}
	}
\end{lstlisting}

\section{Statusy HTTP}

\subsection{Statusy sukcesu}
\begin{table}[h]
	\begin{tabular}{|l|l|l|}
		\hline
		\textbf{Kod} & \textbf{Znaczenie} & \textbf{Przykład} \\
		\hline
		200 & OK & Poprawne pobranie danych \\
		201 & Created & Nowy zasób utworzony \\
		204 & No Content & Operacja udana, brak danych w odpowiedzi \\
		\hline
	\end{tabular}
\end{table}

\subsection{Statusy błędów}
\begin{table}[h]
	\begin{tabular}{|l|l|l|}
		\hline
		\textbf{Kod} & \textbf{Znaczenie} & \textbf{Przykład} \\
		\hline
		400 & Bad Request & Zła składnia lub brakujące pole \\
		401 & Unauthorized & Brak autoryzacji (brak JWT tokena) \\
		403 & Forbidden & Brak uprawnień (user nie jest adminem) \\
		404 & Not Found & Zasób nie istnieje \\
		409 & Conflict & Konflikt danych \\
		500 & Server Error & Wyjątek w kodzie \\
		\hline
	\end{tabular}
\end{table}

\section{DTO i modele}

\subsection{Definicja DTO}
DTO (Data Transfer Object) to obiekt wymiany danych między API a klientem.

\subsection{Przykład}
\begin{lstlisting}
	public class CityDto
	{
		public int Id { get; set; }
		public string Name { get; set; }
		public string Country { get; set; }
	}
\end{lstlisting}

\subsection{Dlaczego DTO?}
\begin{itemize}
	\item Oddzielenie API od logiki biznesowej
	\item Bezpieczeństwo (nie wysyłamy całej encji z bazą)
	\item Elastyczność (łatwa zmiana struktury)
	\item Walidacja danych
\end{itemize}

\section{Schematy i Swagger}

\subsection{Generowanie dokumentacji}
Swagger automatycznie generuje \texttt{swagger.json} na \texttt{/swagger/v1/swagger.json}

\subsection{Wykorzystanie schematu}
\begin{itemize}
	\item Dokumentacja API dla frontend developerów
	\item Generowanie klienta automatycznie (NSwag, OpenAPI Generator)
	\item Walidacja zapytań
\end{itemize}

\section{Wysyłanie zapytań z aplikacji klienckiej}

\subsection{Konsola - przykład}
\begin{lstlisting}
	dotnet new console -n ShopAPI. Client
	cd ShopAPI.Client
	dotnet add package System.Net.Http.Json
	
	// Program.cs
	using System. Net.Http. Json;
	
	var client = new HttpClient();
	client.BaseAddress = new Uri("https://localhost:7294");
	
	// GET
	var cities = await client.GetFromJsonAsync<List<CityDto>>(
	"/api/cities");
	Console.WriteLine($"Miasta: {string.Join(", ", 
		cities.Select(c => c.Name))}");
	
	// POST
	var newCity = new CityDto 
	{ 
		Name = "Warsaw", 
		Country = "Poland" 
	};
	var response = await client.PostAsJsonAsync(
	"/api/cities", newCity);
	Console.WriteLine($"Dodano miasto, status: {response.StatusCode}");
\end{lstlisting}

\section{Autentykacja i autoryzacja}

\subsection{Flow OAuth 2.0 + PKCE}
\begin{enumerate}
	\item Użytkownik kliknie "Zaloguj się"
	\item Aplikacja generuje \texttt{code\_challenge} i \texttt{state}
	\item Redirect do providera (Google / Microsoft Entra ID)
	\item Provider zwraca \texttt{code}
	\item Aplikacja wymienia \texttt{code} na \texttt{access\_token}
	\item Aplikacja wysyła \texttt{access\_token} w nagłówku \texttt{Authorization: Bearer}
	\item API weryfikuje token i zwraca dane
\end{enumerate}

\subsection{Google Cloud Console}
\begin{enumerate}
	\item Utwórz OAuth consent screen
	\item Dodaj zakresy: \texttt{openid, email, profile}
	\item Utwórz OAuth Client ID
	\item Zanotuj: Client ID, Client Secret, Redirect URI
\end{enumerate}

\subsection{Microsoft Entra ID}
\begin{enumerate}
	\item Azure Portal → App registrations
	\item Zanotuj: Application ID, Tenant ID
	\item Dodaj Redirect URIs
	\item Włącz "ID tokens"
\end{enumerate}

\subsection{Integracja w . NET API}
\begin{lstlisting}
	builder
	.AddAuthentication(JwtBearer)
	.AddJwtBearer(options =>
	{
		options.Authority = 
		"https://login.microsoftonline.com/{tenant}/v2.0";
		options. Audience = "{client-id}";
	});
	
	[Authorize]
	[HttpGet]
	public ActionResult<List<CityDto>> GetAll() 
	=> Ok(cities);
\end{lstlisting}

\section{Paginacja i filtrowanie}

\subsection{Przykład}
\begin{lstlisting}
	// GET /api/cities? page=1&pageSize=10&country=Poland
	[HttpGet]
	public ActionResult<PagedResult<CityDto>> GetAll(
	int page = 1, 
	int pageSize = 10, 
	string?  country = null)
	{
		var query = cities.AsQueryable();
		
		if (!string.IsNullOrEmpty(country))
		query = query.Where(c => c.Country == country);
		
		var total = query.Count();
		var items = query
		.Skip((page - 1) * pageSize)
		. Take(pageSize)
		. ToList();
		
		return Ok(new PagedResult<CityDto>
		{
			Items = items,
			Total = total,
			Page = page,
			PageSize = pageSize
		});
	}
\end{lstlisting}

\section{Walidacja danych}

\subsection{Data Annotations}
\begin{lstlisting}
	public class CityDto
	{
		[Required(ErrorMessage = "Imię jest wymagane")]
		[MinLength(2)]
		public string Name { get; set; }
		
		[Required]
		public string Country { get; set; }
	}
\end{lstlisting}

\subsection{Walidacja automatyczna}
ASP.NET automatycznie waliduje na podstawie atrybutów i zwraca 400 Bad Request z szczegółami błędów.

\section{Pytania kontrolne}

\begin{enumerate}
	\item Jakie są podstawowe metody HTTP i ich zastosowania?
	\item Kiedy używać PUT, a kiedy PATCH?
	\item Jaka jest różnica między 200 a 201 a 204?
	\item Co to jest DTO i dlaczego go używamy?
	\item Jak obsługiwać błędy w API?
	\item Jakie są kroki OAuth 2.0 + PKCE flow?
	\item Jak zaimplementować autoryzację w API?
	\item Jak dodać paginację do endpointu?
\end{enumerate}
	\newpage
	\chapter{Tworzenie aplikacji MVC i MVVM}

\section{Wstęp}
MVC (Model-View-Controller) i MVVM (Model-View-ViewModel) to wzorce architektoniczne zapewniające separację warstw i ułatwiające testowanie oraz utrzymanie kodu.

\section{Dlaczego architektura w . NET MAUI?}

\subsection{Problemy bez architektury}
\begin{itemize}
	\item Logika biznesowa mieszana z UI (code-behind)
	\item Brak możliwości testów jednostkowych
	\item Trudna rozbudowa i utrzymanie
	\item Duplikacja kodu
\end{itemize}

\subsection{Korzyści}
\begin{enumerate}
	\item Oddzielenie warstw (UI, logika prezentacji, logika biznesowa)
	\item ViewModel bez zależności od UI - łatwe testowanie
	\item Czystszy kod, mniejsza duplikacja
	\item Łatwiejsze utrzymanie i rozwój
\end{enumerate}

\section{MVC vs MVVM}

\subsection{MVC (ASP.NET Core)}
\begin{description}
	\item[Model] Dane aplikacji, baza danych
	\item[View] HTML/Razor, interfejs użytkownika
	\item[Controller] Pośrednik, obsługuje żądania HTTP
\end{description}

\subsubsection{Flow MVC}
\begin{enumerate}
	\item Użytkownik wysyła żądanie HTTP
	\item Controller odbiera żądanie
	\item Controller pobrania dane z Model
	\item Controller zwraca View (HTML)
\end{enumerate}

\subsection{MVVM (WPF / MAUI)}
\begin{description}
	\item[Model] Dane aplikacji
	\item[View] XAML, interfejs użytkownika
	\item[ViewModel] Stan + logika prezentacji + INotifyPropertyChanged
\end{description}

\subsubsection{Flow MVVM}
\begin{enumerate}
	\item Użytkownik klika przycisk w View
	\item Command w ViewModel obsługuje zdarzenie
	\item ViewModel zmienia stan (właściwości)
	\item Binding automatycznie aktualizuje View
\end{enumerate}

\section{Data Binding}

\subsection{Tryby bindingu}
\begin{table}[h]
	\begin{tabular}{|l|l|l|}
		\hline
		\textbf{Tryb} & \textbf{Kierunek} & \textbf{Opis} \\
		\hline
		OneWay & View ← ViewModel & Tylko zmiana danych w VM \\
		TwoWay & View ↔ ViewModel & Bidirektjonalny przepływ \\
		OneTime & View ← ViewModel & Binding tylko raz przy załadowaniu \\
		\hline
	\end{tabular}
\end{table}

\subsection{Compiled Bindings}
\begin{lstlisting}
	<!-- XAML -->
	<ContentPage
	xmlns="http://schemas.microsoft.com/dotnet/2021/maui"
	x:DataType="local:MainViewModel">
	
	<Entry Text="{Binding FirstName, Mode=TwoWay}" />
	<Label Text="{Binding FullName}" />
	<Button Command="{Binding SaveCommand}" Text="Zapisz" />
	</ContentPage>
\end{lstlisting}

\subsection{Benefit x:DataType}
\begin{itemize}
	\item Lepsze IntelliSense
	\item Lepsza wydajność (compiled bindings)
	\item Walidacja w compile time
\end{itemize}

\section{CommunityToolkit.Mvvm}

\subsection{Instalacja}
\begin{lstlisting}
	dotnet add package CommunityToolkit.Mvvm
\end{lstlisting}

\subsection{ObservableObject}
Klasa bazowa dla ViewModel z wbudowanym INotifyPropertyChanged. 

\subsection{ObservableProperty}
\begin{lstlisting}
	public partial class MainViewModel : ObservableObject
	{
		[ObservableProperty]
		private string firstName;
		
		[ObservableProperty]
		private string lastName;
		
		public string FullName => $"{FirstName} {LastName}";
	}
\end{lstlisting}

\subsection{Source Generators}
Toolkit automatycznie generuje:
\begin{itemize}
	\item PropertyChanged event
	\item Pełne implementacje właściwości
	\item RelayCommand handlers
\end{itemize}

\section{Commands vs Event Handlers}

\subsection{RelayCommand}
\begin{lstlisting}
	public partial class MainViewModel : ObservableObject
	{
		[RelayCommand]
		private async Task Save()
		{
			IsBusy = true;
			try
			{
				await Task.Delay(1000); // Symulacja
			}
			finally
			{
				IsBusy = false;
			}
		}
	}
	
	// W XAML
	<Button Command="{Binding SaveCommand}" Text="Zapisz" />
\end{lstlisting}

\subsection{CanExecute}
\begin{lstlisting}
	public partial class MainViewModel : ObservableObject
	{
		[ObservableProperty]
		private bool isBusy;
		
		[RelayCommand(CanExecute = nameof(IsNotBusy))]
		private async Task Save()
		{
			// ... 
		}
		
		private bool IsNotBusy => !IsBusy;
	}
\end{lstlisting}

\subsection{ICommand vs Event Handlers}
\begin{table}[h]
	\begin{tabular}{|l|l|}
		\hline
		\textbf{ICommand} & \textbf{Event Handlers} \\
		\hline
		Testowalne & Trudne do testowania \\
		CanExecute (enable/disable) & Brak warunkowego disablowania \\
		Data binding & Require code-behind \\
		MVVM friendly & Tight coupling \\
		\hline
	\end{tabular}
\end{table}

\section{Model vs DTO vs Entity}

\subsection{Entity}
Klasa reprezentująca wiersz w bazie danych, zawiera ID. 

\subsection{DTO}
Obiekt transferu danych między warstwami, odseparowany od bazy.

\subsection{Model}
Obiekt używany w aplikacji, może być kombinacją Entity i DTO.

\subsection{Przykład}
\begin{lstlisting}
	// Entity
	public class User
	{
		public int Id { get; set; }
		public string Email { get; set; }
		public string PasswordHash { get; set; }
	}
	
	// DTO
	public class UserDto
	{
		public int Id { get; set; }
		public string Email { get; set; }
	}
	
	// ViewModel
	public class UserViewModel : ObservableObject
	{
		[ObservableProperty]
		private string email;
		
		public void LoadFrom(UserDto dto)
		{
			Email = dto.Email;
		}
	}
\end{lstlisting}

\section{Walidacja}

\subsection{ValidationAttributes}
\begin{lstlisting}
	public class ContactViewModel : ObservableObject
	{
		[ObservableProperty]
		[Required(ErrorMessage = "Imię jest wymagane")]
		[MinLength(2, ErrorMessage = "Min 2 znaki")]
		private string firstName;
		
		[ObservableProperty]
		private string validationMessage;
		
		[RelayCommand]
		private void Save()
		{
			var context = new ValidationContext(this);
			var results = new List<ValidationResult>();
			
			if (! Validator.TryValidateObject(this, context, 
			results, true))
			{
				ValidationMessage = string.Join("\n", 
				results.Select(r => r.ErrorMessage));
			}
		}
	}
\end{lstlisting}

\section{ValueConverter}

\subsection{Przykład - BoolInverseConverter}
\begin{lstlisting}
	public class BoolInverseConverter : IValueConverter
	{
		public object Convert(object value, Type targetType, 
		object parameter, CultureInfo culture)
		{
			return !(bool)value;
		}
		
		public object ConvertBack(object value, Type targetType, 
		object parameter, CultureInfo culture)
		{
			return !(bool)value;
		}
	}
	
	// ResourceDictionary
	<ResourceDictionary 
	xmlns="http://schemas.microsoft.com/dotnet/2021/maui"
	xmlns:local="clr-namespace:MyApp">
	<local:BoolInverseConverter x:Key="BoolInverse" />
	</ResourceDictionary>
	
	// XAML
	<Button Text="Zapisz" 
	IsEnabled="{Binding IsBusy, 
		Converter={StaticResource BoolInverse}}" />
\end{lstlisting}

\section{Dependency Injection}

\subsection{MAUI}
\begin{lstlisting}
	// MauiProgram.cs
	var builder = MauiApp.CreateBuilder();
	builder
	.UseMauiApp<App>()
	.ConfigureFonts(fonts =>
	{
		fonts.AddFont("OpenSans-Regular.ttf", "OpenSansRegular");
	})
	.ConfigureServices(services =>
	{
		services.AddSingleton<MainViewModel>();
		services.AddTransient<MainPage>();
	});
	
	// MainPage.xaml. cs
	public MainPage(MainViewModel vm)
	{
		InitializeComponent();
		BindingContext = vm;
	}
\end{lstlisting}

\section{Testy jednostkowe}

\subsection{Projekt testowy}
\begin{lstlisting}
	dotnet new nunit -n DemoTests
	dotnet add DemoTests/DemoTests.csproj reference \
	DemoMaui/DemoMaui. csproj
\end{lstlisting}

\subsection{Przykładowy test}
\begin{lstlisting}
	[TestFixture]
	public class MainViewModelTests
	{
		[Test]
		public void FullName_Updates_OnFirstOrLastChange()
		{
			var vm = new MainViewModel();
			vm.FirstName = "Anna";
			vm.LastName = "Nowak";
			
			Assert. That(vm.FullName, Is.EqualTo("Anna Nowak"));
		}
		
		[Test]
		public async Task Save_SetsBusyTrue()
		{
			var vm = new MainViewModel();
			
			Assert.That(vm.IsBusy, Is.False);
			
			var task = vm.SaveCommand.ExecuteAsync(null);
			
			Assert.That(vm.IsBusy, Is.True);
			
			await task;
			
			Assert.That(vm.IsBusy, Is.False);
		}
	}
\end{lstlisting}

\section{Nawigacja}

\subsection{MAUI Shell}
\begin{lstlisting}
	// AppShell.xaml. cs
	public partial class AppShell : Shell
	{
		public AppShell()
		{
			InitializeComponent();
			Routing.RegisterRoute("details", typeof(DetailsPage));
		}
	}
	
	// MainViewModel.cs
	[RelayCommand]
	private async Task SelectContact(Contact contact)
	{
		await Shell.Current.GoToAsync(
		$"details?id={contact.Id}");
	}
	
	// DetailsPage.xaml. cs
	[QueryProperty(nameof(ContactId), "id")]
	public partial class DetailsPage : ContentPage
	{
		private int contactId;
		public int ContactId
		{
			get => contactId;
			set => contactId = value;
		}
	}
\end{lstlisting}

\section{Stylizacja}

\subsection{ResourceDictionary}
\begin{lstlisting}
	<!-- Resources/Styles. xaml -->
	<ResourceDictionary
	xmlns="http://schemas.microsoft.com/dotnet/2021/maui">
	
	<Color x:Key="PrimaryColor">#512BD4</Color>
	<Color x:Key="SecondaryColor">#DFD8F7</Color>
	
	<Style TargetType="Button" ApplyToDerivedTypes="True">
	<Setter Property="BackgroundColor" 
	Value="{StaticResource PrimaryColor}" />
	<Setter Property="Padding" Value="10,5" />
	<Setter Property="FontSize" Value="14" />
	</Style>
	</ResourceDictionary>
\end{lstlisting}

\subsection{Light/Dark Mode}
\begin{lstlisting}
	<Label Text="Hello" 
	TextColor="{AppThemeBinding Light=Black, 
		Dark=White}" />
\end{lstlisting}

\section{MAUI vs WPF}

\subsection{Porównanie}
\begin{table}[h]
	\begin{tabular}{|l|l|l|}
		\hline
		\textbf{Aspekt} & \textbf{MAUI} & \textbf{WPF} \\
		\hline
		Platformy & Android, iOS, Windows, macOS & Windows \\
		Projekt & Single Project & Classic Desktop \\
		Handlers & Handler architecture & Dependency Properties \\
		Hot Reload & Wieloplatformowy & Desktopowy \\
		\hline
	\end{tabular}
\end{table}

\subsection{Lifecycle MAUI}
\begin{itemize}
	\item \textbf{OnStart}: Aplikacja uruchomiona
	\item \textbf{OnSleep}: Aplikacja przechodzi w tło
	\item \textbf{OnResume}: Aplikacja wznowiona z tła
\end{itemize}

\subsection{Uprawnienia na mobilnych}
\begin{lstlisting}
	// AndroidManifest.xml
	<uses-permission android:name="android.permission. CAMERA" />
	
	// Kod
	var status = await Permissions.CheckStatusAsync<Permissions.Camera>();
	if (status != PermissionStatus.Granted)
	{
		status = await Permissions.RequestAsync<Permissions.Camera>();
	}
\end{lstlisting}

\section{Pytania kontrolne}

\begin{enumerate}
	\item Jaka jest różnica między MVC a MVVM?
	\item Co to jest Data Binding i jakie są jego tryby?
	\item Jak działa ObservableProperty w MVVM Toolkit?
	\item Dlaczego RelayCommand jest lepszy niż event handler?
	\item Co to jest DTO i dlaczego oddzielamy go od Entity?
	\item Jak zaimplementować walidację w MVVM?
	\item Jakie są korzyści z Dependency Injection?
	\item Jakie są główne różnice między MAUI a WPF?
	\item Jak obsługiwać uprawnienia na urządzeniach mobilnych?
\end{enumerate}
	\newpage
	\chapter{Tworzenie aplikacji Blazor}

\section{Wstęp}
Blazor to framework do tworzenia aplikacji webowych w .NET.  Umożliwia tworzenie interaktywnych UI przy użyciu C# zamiast JavaScript. 

\section{Dlaczego Blazor?}

\subsection{Korzyści}
\begin{itemize}
	\item \textbf{Jeden zablon}: Obsługuje wiele trybów renderowania
	\item \textbf{SEO + szybkość}: SSR dla pierwszego ładowania
	\item \textbf{Wspólny kod}: Model, walidacja, logika w . NET (brak duplikacji JS/TS)
	\item \textbf{Progressive Enhancement}: Stopniowe dodawanie interaktywności
\end{itemize}

\section{Tryby renderowania}

\subsection{Static SSR (Server-Side Rendering)}
\begin{description}
	\item[Opis] Strona jest renderowana na serwerze, brak interaktywności
	\item[Użycie] Content, marketing, blog, landing page
	\item[Zalety] Minimalny rozmiar, SEO, szybkie ładowanie
	\item[Wady] Brak interaktywności
\end{description}

\begin{lstlisting}
	// Program.cs - nie dodajemy interactive
	var builder = WebApplication.CreateBuilder(args);
	builder.Services.AddRazorComponents();
	
	var app = builder.Build();
	app.MapRazorComponents<App>();
	app.Run();
\end{lstlisting}

\subsection{Interactive Server}
\begin{description}
	\item[Opis] Logika po stronie serwera, komunikacja przez WebSocket
	\item[Użycie] Backoffice, aplikacje z małą ilością użytkowników
	\item[Zalety] Mały transfer, szybkie TTFB, pełna moc serwera
	\item[Wady] Połączenia stałe, ograniczona skalowalność
\end{description}

\begin{lstlisting}
	// Component.razor
	@rendermode InteractiveServer
	
	@page "/counter"
	
	<h1>Counter</h1>
	<p>Current count: @count</p>
	<button @onclick="IncrementCount">Click me</button>
	
	@code {
		private int count = 0;
		
		private void IncrementCount()
		{
			count++;
		}
	}
\end{lstlisting}

\subsection{Interactive WebAssembly}
\begin{description}
	\item[Opis] Logika w przeglądarce, brak zależności od serwera
	\item[Użycie] Aplikacje wymagające offline, PWA
	\item[Zalety] Skalowalność, offline, PWA, brak roundtrip
	\item[Wady] Większy initial payload, cold start
\end{description}

\begin{lstlisting}
	// Program.cs - dodajemy WebAssembly
	builder.Services.AddRazorComponents()
	.AddInteractiveWebAssemblyComponents();
\end{lstlisting}

\subsection{Interactive Auto}
\begin{description}
	\item[Opis] Pierwszy raz Server (szybki), potem przełącza na WebAssembly
	\item[Użycie] Aplikacje wymagające szybkiego TTFB i skalowalności
	\item[Zalety] Najlepsze z obu światów
	\item[Wady] Większa złożoność
\end{description}

\begin{lstlisting}
	@rendermode InteractiveAuto
\end{lstlisting}

\section{Stream Rendering}

\subsection{Problem bez stream renderingu}
Użytkownik widzi białą stronę, aż wszystkie dane będą pobrane.

\subsection{Stream rendering - rozwiązanie}
\begin{enumerate}
	\item Szkielet strony zaraz
	\item Placeholder "Ładowanie..."
	\item Dane wypływają stopniowo
\end{enumerate}

\subsection{Implementacja}
\begin{lstlisting}
	@page "/products"
	@rendermode InteractiveServer
	
	<h1>Products</h1>
	
	@if (products == null)
	{
		<p>Ładowanie...</p>
	}
	else
	{
		@foreach (var product in products)
		{
			<p>@product.Name</p>
		}
	}
	
	@code {
		private List<Product> products;
		
		protected override async Task OnInitializedAsync()
		{
			await Task.Delay(1500); // Symulacja
			products = new List<Product>
			{
				new Product { Name = "Product 1" },
				new Product { Name = "Product 2" }
			};
		}
	}
\end{lstlisting}

\section{Struktura projektu Blazor Web App}

\subsection{Komponenty}
\begin{description}
	\item[Server] Host, API, SSR
	\item[Client] WebAssembly (komponenty interaktywne)
	\item[Shared] Modele, DTO, interfejsy usług
\end{description}

\subsection{Program. cs - Server}
\begin{lstlisting}
	var builder = WebApplicationBuilder.CreateBuilder(args);
	
	builder.Services
	.AddRazorComponents()
	.AddInteractiveServerComponents()
	.AddInteractiveWebAssemblyComponents();
	
	builder.Services.AddScoped<GameService>();
	
	var app = builder.Build();
	
	app.UseStaticFiles();
	app.UseRouting();
	app.MapRazorComponents<App>()
	.AddInteractiveServerRenderMode()
	.AddInteractiveWebAssemblyRenderMode();
	
	app.Run();
\end{lstlisting}

\section{Formularze}

\subsection{EditForm}
\begin{lstlisting}
	@page "/contacts/edit"
	@rendermode InteractiveServer
	
	<h1>Nowy kontakt</h1>
	
	<EditForm Model="contact" OnValidSubmit="HandleSubmit">
	<DataAnnotationsValidator />
	<ValidationSummary />
	
	<div>
	<label>Imię:</label>
	<InputText @bind-Value="contact.FirstName" />
	</div>
	
	<div>
	<label>Email:</label>
	<InputEmail @bind-Value="contact.Email" />
	</div>
	
	<button type="submit">Zapisz</button>
	</EditForm>
	
	@code {
		private Contact contact = new();
		
		private async Task HandleSubmit()
		{
			await contactService.SaveAsync(contact);
		}
	}
\end{lstlisting}

\subsection{SSR static formularze}
\begin{lstlisting}
	@page "/contact"
	
	<h1>Kontakt</h1>
	
	<form method="post">
	<input type="hidden" name="FormName" value="Contact" />
	
	<div>
	<label>Imię:</label>
	<input type="text" name="firstName" required />
	</div>
	
	<button type="submit">Wyślij</button>
	</form>
	
	@code {
		[SupplyParameterFromForm(FormName = "Contact")]
		private string FirstName { get; set; }
		
		public async Task OnPostAsync()
		{
			if (FirstName != null)
			{
				// Przetwórz dane
			}
		}
	}
\end{lstlisting}

\section{Walidacja}

\subsection{Data Annotations}
\begin{lstlisting}
	public class Contact
	{
		[Required(ErrorMessage = "Imię jest wymagane")]
		[MinLength(2, ErrorMessage = "Min 2 znaki")]
		public string FirstName { get; set; }
		
		[Required]
		[EmailAddress]
		public string Email { get; set; }
	}
\end{lstlisting}

\subsection{Custom validator}
\begin{lstlisting}
	public class CustomValidator : ComponentBase
	{
		private EditContext editContext;
		
		[CascadingParameter]
		private EditContext EditContext
		{
			get => editContext;
			set
			{
				editContext = value;
				editContext?.AddAsyncValidator(this);
			}
		}
		
		public async Task ValidateAsync(
		ValidationContext context)
		{
			// Custom validacja
		}
	}
\end{lstlisting}

\section{EF Core w Blazor}

\subsection{DbContext}
\begin{lstlisting}
	public class GameContext : DbContext
	{
		public DbSet<Game> Games { get; set; }
		
		protected override void OnConfiguring(
		DbContextOptionsBuilder options)
		{
			options. UseSqlServer(
			"Server=localhost;Database=Games;.. .");
		}
		
		protected override void OnModelCreating(
		ModelBuilder modelBuilder)
		{
			modelBuilder.Entity<Game>(). HasData(
			new Game { Id = 1, Title = "Elden Ring" },
			new Game { Id = 2, Title = "Baldur's Gate 3" }
			);
		}
	}
\end{lstlisting}

\subsection{Migracje}
\begin{lstlisting}
	dotnet ef migrations add InitialCreate
	dotnet ef database update
\end{lstlisting}

\subsection{Rejestracja w DI}
\begin{lstlisting}
	builder.Services.AddDbContext<GameContext>(options =>
	options.UseSqlServer(
	builder.Configuration.GetConnectionString("Default")));
\end{lstlisting}

\section{Service Layer}

\subsection{Server - bezpośredni dostęp do DbContext}
\begin{lstlisting}
	public interface IGameService
	{
		Task<List<Game>> GetAllAsync();
		Task<Game> GetByIdAsync(int id);
		Task CreateAsync(Game game);
	}
	
	public class GameService : IGameService
	{
		private readonly GameContext context;
		
		public GameService(GameContext context)
		{
			this.context = context;
		}
		
		public async Task<List<Game>> GetAllAsync()
		{
			return await context.Games.ToListAsync();
		}
		
		public async Task CreateAsync(Game game)
		{
			context.Games.Add(game);
			await context.SaveChangesAsync();
		}
	}
\end{lstlisting}

\subsection{Client - HttpClient}
\begin{lstlisting}
	public class ClientGameService : IGameService
	{
		private readonly HttpClient httpClient;
		
		public ClientGameService(HttpClient httpClient)
		{
			this.httpClient = httpClient;
		}
		
		public async Task<List<Game>> GetAllAsync()
		{
			return await httpClient.GetFromJsonAsync<List<Game>>(
			"/api/games");
		}
		
		public async Task CreateAsync(Game game)
		{
			await httpClient.PostAsJsonAsync("/api/games", game);
		}
	}
\end{lstlisting}

\section{Komponenty CRUD}

\subsection{Lista z stream renderingiem}
\begin{lstlisting}
	@page "/games"
	@rendermode InteractiveServer
	@inject IGameService gameService
	
	<h1>Games</h1>
	
	<button @onclick="NavigateToCreate">Add New</button>
	
	@if (games == null)
	{
		<p>Loading...</p>
	}
	else
	{
		<table>
		<thead>
		<tr>
		<th>Title</th>
		<th>Actions</th>
		</tr>
		</thead>
		<tbody>
		@foreach (var game in games)
		{
			<tr>
			<td>@game.Title</td>
			<td>
			<button @onclick='() => NavigateToEdit(game. Id)'>
			Edit
			</button>
			<button @onclick='() => Delete(game.Id)'>
			Delete
			</button>
			</td>
			</tr>
		}
		</tbody>
		</table>
	}
	
	@code {
		private List<Game> games;
		
		protected override async Task OnInitializedAsync()
		{
			games = await gameService.GetAllAsync();
		}
		
		private void NavigateToCreate()
		{
			// Navigate to create page
		}
		
		private async Task Delete(int id)
		{
			await gameService.DeleteAsync(id);
			games = await gameService.GetAllAsync();
		}
	}
\end{lstlisting}

\subsection{Edycja - Auto render mode}
\begin{lstlisting}
	@page "/games/edit"
	@page "/games/edit/{id:int}"
	@rendermode InteractiveAuto
	@inject IGameService gameService
	
	<h1>@(game.Id == 0 ? "Add" : "Edit") Game</h1>
	
	<EditForm Model="game" OnValidSubmit="Save">
	<DataAnnotationsValidator />
	<ValidationSummary />
	
	<InputText @bind-Value="game.Title" 
	placeholder="Title" />
	
	<button type="submit">Save</button>
	</EditForm>
	
	@code {
		[Parameter]
		public int Id { get; set; }
		
		private Game game = new();
		
		protected override async Task OnInitializedAsync()
		{
			if (Id > 0)
			{
				game = await gameService.GetByIdAsync(Id);
			}
		}
		
		private async Task Save()
		{
			if (game.Id == 0)
			await gameService.CreateAsync(game);
			else
			await gameService.UpdateAsync(game);
		}
	}
\end{lstlisting}

\section{PWA (WebAssembly)}

\subsection{Service Worker}
\begin{itemize}
	\item Cache zasobów offline
	\item Synchronizacja w tle
	\item Push notyfikacje
\end{itemize}

\subsection{Włączenie PWA}
Blazor WebAssembly automatycznie obsługuje PWA.  Dodaj manifest i service worker. 

\section{Pre-rendering}

\subsection{Włączone pre-rendering}
\begin{lstlisting}
	@rendermode new InteractiveServerRenderMode(prerender: true)
\end{lstlisting}

\subsection{Wyłączone pre-rendering}
\begin{lstlisting}
	@rendermode new InteractiveServerRenderMode(prerender: false)
\end{lstlisting}

\section{Pytania kontrolne}

\begin{enumerate}
	\item Jakie są główne tryby renderowania w Blazor? 
	\item Kiedy wybrać Static SSR, a kiedy Interactive? 
	\item Co to jest stream rendering i jakie są jego korzyści?
	\item Jak zaimplementować formularz w Blazor? 
	\item Jakie są różnice między EditForm a HTML form?
	\item Jak zintegrować EF Core z Blazor? 
	\item Co to jest PWA i kiedy go używać?
	\item Jaka jest różnica między pre-rendering włączonym a wyłączonym?
	\item Jak obsługiwać walidację w formularzach?
\end{enumerate}
	\newpage
	\chapter{Tworzenie aplikacji rozproszonych - .  NET Aspire}

\section{Wstęp}
.  NET Aspire to framework do tworzenia i zarządzania aplikacjami rozproszonymi.  Upraszcza orkestrację wielu usług, obsługuje monitoring i deployment. 

\section{Problemy przy wielu usługach}

\subsection{Bez Aspire}
\begin{itemize}
	\item Ręczne uruchamianie wielu projektów
	\item Różne porty i konfiguracje u różnych developerów
	\item Brak spójnej obserwowalności
	\item Ręczne tworzenie connection stringów
	\item Ręczne uruchamianie baz danych (Docker)
	\item Różne konfiguracje DEV vs PROD
\end{itemize}

\subsection{Z Aspire}
\begin{itemize}
	\item Deklaratywny opis zasobów (AppHost)
	\item Service Discovery (identyfikatory zamiast URL)
	\item Integracje (Redis, PostgreSQL, Service Bus)
	\item Automatyczne health checks i telemetria
	\item Resilience (retry, timeout, circuit breaker)
	\item Dashboard do monitorowania
	\item Wdrożenie na Azure Container Apps lub Kubernetes
\end{itemize}

\section{Kluczowe pojęcia}

\subsection{AppHost}
Projekt, który definiuje wszystkie zasoby i ich relacje. 

\subsection{Service Defaults}
Wspólne rozszerzenia dla wszystkich usług.

\subsection{Service Discovery}
Automatyczne odkrywanie usług bez podawania portów.

\section{Health Checks}

\subsection{/health}
Zwraca zdrowie usługi razem z zależnościami (baza, cache). 

\subsection{/alive}
Zwraca tylko, czy proces żyje.

\section{Telemetria}

\subsection{Logs}
Ustrukturyzowane logi zapisywane w Dashboard.

\subsection{Distributed Traces}
Śledzenie żądania przez wiele usług.

\subsection{Metrics}
HTTP metrics, GC stats, custom metrics.

\section{Resilience - Polly}

\subsection{Domyślnie wbudowane}
\begin{itemize}
	\item Retry z jitterem
	\item Timeout
	\item Circuit breaker
\end{itemize}

\section{Komponenty}

\subsection{Redis}
Pamięć podręczna dla rozproszonego cache'a.

\subsection{PostgreSQL}
Baza danych dla aplikacji rozproszonej.

\subsection{Service Bus}
Komunikacja między usługami.

\section{Pytania kontrolne}

\begin{enumerate}
	\item Jakie są główne problemy przy zarządzaniu wieloma usługami?
	\item Co to jest AppHost w .  NET Aspire?
	\item Jak Service Discovery ułatwia zarządzanie usługami?
	\item Jakie są różnice między /health a /alive?
	\item Jakie metryki zbiera Telemetria?
	\item Co to jest Polly i jakie strategie oferuje?
	\item Jak integrować Redis z .  NET Aspire?
	\item Jak monitorować aplikacje rozproszone?
\end{enumerate}
	
\end{document}