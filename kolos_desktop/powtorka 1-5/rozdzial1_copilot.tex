\chapter{Programowanie AI - Copilot, VS Code}

\section{Wstęp}
GitHub Copilot to asystent AI w Visual Studio Code, który wspomaga pracę programisty.  Zajęcia skupiają się na zaawansowanych funkcjach, od Agent Mode po personalizację poprzez instrukcje. 

\section{Agent Mode i jego zastosowania}

\subsection{Definicja}
Agent Mode to tryb, w którym Copilot działa automatycznie bez konieczności potwierdzania każdego kroku. Różni się od klasycznego trybu interaktywnego. 

\subsection{Praktyczne zastosowania}
\begin{itemize}
	\item \textbf{Build i test}: automatyczne budowanie i testowanie projektów
	\item \textbf{Długotrwałe zadania}: delegowanie złożonych operacji
	\item \textbf{Code review}: analiza kodu przez agenta
	\item \textbf{Generowanie endpointów API}: szybkie tworzenie usług
\end{itemize}

\subsection{Tryby pracy}
\begin{description}
	\item[Interaktywny] Copilot prosi o potwierdzenie każdego kroku
	\item[W pełni automatyczny] Agent wykonuje całe zadanie bez przerw
\end{description}

\section{Allow / Deny List – bezpieczeństwo}

\subsection{Rola listy dozwolonych komend}
Allow List zapewnia bezpieczeństwo przez limitowanie komend, które agent może wykonać automatycznie.

\subsection{Praktyczne przykłady}
\begin{lstlisting}
	// Dozwolone komendy
	- dotnet build
	- dotnet test
	- dotnet run
	
	// Zakazane komendy
	- rm -rf . 
	- del *.*
\end{lstlisting}

\subsection{Konfiguracja}
\begin{enumerate}
	\item Otwórz Settings → Copilot → Experimental
	\item Dodaj komendy do Allow List:
	\begin{itemize}
		\item \texttt{dotnet build}
		\item \texttt{dotnet test}
	\end{itemize}
	\item Po skonfigurowaniu projekt buduje się automatycznie bez pytania
\end{enumerate}

\section{Zarządzanie limitami - Max Requests}

\subsection{Wpływ na wydajność}
Domyślnie Max Requests wynosi 20.  Limit wpływa na:
\begin{itemize}
	\item Ilość automatycznych operacji
	\item Płynność pracy z agentem
	\item Czas wykonywania długotrwałych zadań
\end{itemize}

\subsection{Praktyczne ustawienia}
\begin{table}[h]
	\begin{tabular}{|l|l|l|}
		\hline
		\textbf{Projekt} & \textbf{Max Requests} & \textbf{Uzasadnienie} \\
		\hline
		Mały projekt & 20 & Domyślnie wystarczy \\
		Średni projekt & 50-100 & Lepszy flow pracy \\
		Duży projekt & 100+ & Złożone zadania \\
		\hline
	\end{tabular}
\end{table}

\section{Copilot Instructions - personalizacja}

\subsection{Automatyczne generowanie}
\begin{enumerate}
	\item Otwórz chat Copilota
	\item Kliknij "Customize chat" → "Generate Instructions"
	\item Copilot skanuje projekt i tworzy \texttt{copilot-instructions.md}
\end{enumerate}

\subsection{Zawartość wygenerowanych instrukcji}
\begin{itemize}
	\item Struktura projektu
	\item Komendy budowania i uruchamiania
	\item Konwencje kodu (PascalCase, camelCase)
	\item DTO z atrybutami JSON
	\item Reguły CSS
\end{itemize}

\subsection{Przykład własnych reguł}
\begin{lstlisting}
	# Copilot Instructions
	
	## Konwencje kodu
	- Klasy i metody: PascalCase
	- Zmienne: camelCase
	- Pola private: _camelCase
	
	## DTO
	Wszystkie DTO mają atrybuty JSON:
	[JsonPropertyName("fieldName")]
	
	## C# Best Practices
	- Używaj using statements
	- Dokumentuj publiczne metody XML comments
\end{lstlisting}

\subsection{Rozdzielanie zasad dla backendu i frontendu}
Utwórz oddzielne pliki:
\begin{itemize}
	\item \texttt{. github/csharp-guidelines.md}
	\item \texttt{.github/blazor-guidelines.md}
\end{itemize}

\section{Snooze Completions - balans AI/samodzielne kodowanie}

\subsection{Kiedy wyciszyć Copilota}
\begin{itemize}
	\item Nauka nowych konceptów
	\item Ćwiczenie świadomego programowania
	\item Unikanie uzależnienia od AI
\end{itemize}

\subsection{Jak korzystać}
\begin{enumerate}
	\item Kliknij ikonę Copilota w pasku statusu
	\item Wybierz "Snooze" (np. 5 minut)
	\item Pracuj samodzielnie
	\item Po upłynięciu czasu podpowiedzi wracają
\end{enumerate}

\section{Custom Chat Modes}

\subsection{Beast Mode}
Specjalny tryb, w którym AI rozbija zadanie na kroki i realizuje je sekwencyjnie.

\subsection{Konfiguracja}
\begin{enumerate}
	\item Utwórz \texttt{. github/chat-modes/beast.json}
	\item Uruchom Agent Mode
	\item Wybierz Beast Mode z listy trybów
	\item Poproś o dodanie endpointu API
\end{enumerate}

\section{Coding Agents - wirtualny współprogramista}

\subsection{Przypisywanie zadań}
\begin{itemize}
	\item Utwórz issue na GitHub
	\item Przypisz je do @copilot
	\item Agent automatycznie tworzy branch i PR
\end{itemize}

\subsection{Delegowanie pracy}
\begin{enumerate}
	\item W VS Code otwórz panel agenta
	\item Kliknij +, dodaj nowe zadanie
	\item Agent tworzy nowy branch i sesję
	\item Pracuje w tle, podczas gdy ty kodując lokalnie
\end{enumerate}

\subsection{Code review od agenta}
\begin{itemize}
	\item Dodaj Copilota jako reviewera PR
	\item Agent sugeruje ulepszenia
	\item Może proponować użycie ILogger zamiast Console. WriteLine
\end{itemize}

\section{Integracja z GitHub Actions}

\subsection{Automatyzacja setupu środowiska}
Plik \texttt{.github/workflows/copilot-setup-steps.yml}:
\begin{lstlisting}
	name: Setup Steps
	on: [push]
	jobs:
	setup:
	runs-on: ubuntu-latest
	steps:
	- uses: actions/checkout@v3
	- name: Setup .NET 9
	uses: actions/setup-dotnet@v3
	with:
	dotnet-version: '9.0'
	- name: Restore packages
	run: dotnet restore
\end{lstlisting}

\section{Zarządzanie modelami AI}

\subsection{Dostępne modele}
\begin{itemize}
	\item \textbf{GPT-5}: Najnowszy, najlepsze rezultaty
	\item \textbf{GPT-5 mini}: Szybszy, mniej zasobów
	\item \textbf{Starsze modele}: Można usunąć
\end{itemize}

\subsection{Zmiana modelu}
\begin{enumerate}
	\item VS Code 1.103+
	\item Otwórz model selector
	\item Wybierz preferowany model
	\item Usuń starsze modele z listy
\end{enumerate}

\section{MCP Auto Start i narzędzia developerskie}

\subsection{Rola MCP}
MCP (Model Context Protocol) udostępnia dodatkowe narzędzia dla Copilota. 

\subsection{Konfiguracja}
\begin{enumerate}
	\item Settings → MCP Auto Start → ustaw "always"
	\item Configure Tools → przejrzyj dostępne narzędzia
	\item Wyłącz nieużywane, aby zmniejszyć szum
\end{enumerate}

\section{Statystyki użycia AI}

\subsection{Analiza wydajności}
\begin{itemize}
	\item AI vs typing - co naprawdę daje Copilot?
	\item Eksperymenty porównawcze
	\item Eksport danych do analizy
\end{itemize}

\subsection{Interpretacja danych}
\begin{enumerate}
	\item Włącz AI Stats
	\item Sprawdź wykres w dolnym rogu (24h)
	\item Porównaj produktywność z/bez AI
\end{enumerate}

\section{Pytania kontrolne}

\begin{enumerate}
	\item Jaka jest różnica między trybem interaktywnym a w pełni automatycznym?
	\item Dlaczego Allow List jest ważny dla bezpieczeństwa?
	\item Jak wygenerować automatyczne instrukcje dla projektu?
	\item Kiedy warto używać Snooze Completions?
	\item Jak przypisać zadanie do Coding Agent?
	\item Jakie są dostępne modele AI w Copilot?
\end{enumerate}