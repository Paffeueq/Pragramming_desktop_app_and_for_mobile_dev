\section{Teoria: .NET Aspire}

\subsection{Czym jest .NET Aspire? }

\begin{tcolorbox}[colback=blue!5!white,colframe=blue!75!black,title=Definicja]
\textbf{.NET Aspire} to stack technologiczny zaprojektowany do budowy obserwowalnych, gotowych na produkcję aplikacji rozproszonych. Jest dostarczany poprzez zestaw pakietów NuGet, które obsługują konkretne potrzeby aplikacji chmurowych. 
\end{tcolorbox}

\subsubsection{Główne cele . NET Aspire}

\begin{itemize}[leftmargin=*]
    \item \textbf{Orkiestracja} -- Uproszczone zarządzanie wieloma projektami i zależnościami
    \item \textbf{Integracje} -- Łączenie z różnymi usługami (bazy danych, messaging, storage)
    \item \textbf{Narzędzia} -- Wsparcie dla lokalnego developmentu i testowania
    \item \textbf{Telemetria} -- Wbudowane logowanie, metryki i tracing
    \item \textbf{Deployment} -- Gotowe do wdrożenia w środowisku produkcyjnym
\end{itemize}

\subsection{Kluczowe komponenty}

\subsubsection{1. App Host (Orkiestrator)}

\begin{tcolorbox}[colback=green!5!white,colframe=green!75!black,title=App Host]
\textbf{App Host} to projekt orkiestracyjny, który definiuje i konfiguruje wszystkie komponenty aplikacji rozproszonej.
\end{tcolorbox}

\textbf{Charakterystyka:}
\begin{itemize}
    \item Centralny punkt konfiguracji całej aplikacji
    \item Definiuje zależności między serwisami
    \item Zarządza cyklem życia komponentów
    \item Automatycznie konfiguruje Service Discovery
    \item Generuje manifesty dla deployment
\end{itemize}

\textbf{Typowa struktura App Host:}
\begin{lstlisting}[language={[Sharp]C}]
var builder = DistributedApplication.CreateBuilder(args);

// Dodanie zasobow
var cache = builder.AddRedis("cache");
var db = builder.AddPostgres("postgres")
               .AddDatabase("mydb");

// Dodanie serwisow
var apiService = builder.AddProject<Projects.ApiService>("apiservice")
                       .WithReference(cache)
                       .WithReference(db);

builder.AddProject<Projects.Web>("webfrontend")
       .WithReference(apiService);

builder.Build().Run();
\end{lstlisting}

\subsubsection{2. Service Defaults}

\begin{tcolorbox}[colback=yellow! 5!white,colframe=yellow!75!black,title=Service Defaults]
Wspólna konfiguracja dla wszystkich serwisów w aplikacji, zapewniająca spójne ustawienia telemetrii, health checks i resiliency.
\end{tcolorbox}

\textbf{Co zawierają Service Defaults:}
\begin{itemize}
    \item Konfiguracja OpenTelemetry (logging, metrics, tracing)
    \item Health checks
    \item Service Discovery
    \item Standardowe middleware
    \item HTTP client factory z resilience patterns
\end{itemize}

\subsubsection{3.  Service Discovery}

\begin{tcolorbox}[colback=purple!5!white,colframe=purple!75!black,title=Service Discovery]
Automatyczny mechanizm wykrywania i komunikacji między serwisami bez hardcodowania adresów URL.
\end{tcolorbox}

\textbf{Jak działa:}
\begin{enumerate}
    \item App Host rejestruje wszystkie serwisy
    \item Każdy serwis otrzymuje unikalną nazwę
    \item Komunikacja odbywa się przez nazwę serwisu, nie przez URL
    \item W runtime nazwy są rozwiązywane do rzeczywistych adresów
\end{enumerate}

\textbf{Przykład użycia:}
\begin{lstlisting}[language={[Sharp]C}]
// W App Host
var apiService = builder. AddProject<Projects.ApiService>("apiservice");
builder.AddProject<Projects.Web>("webfrontend")
       .WithReference(apiService);

// W serwisie Web
var response = await httpClient.GetAsync("http://apiservice/api/data");
// "apiservice" zostanie automatycznie rozwiazane
\end{lstlisting}

\subsection{Integracje (Components)}

. NET Aspire dostarcza gotowe integracje z popularnymi technologiami:

\begin{table}[h]
\centering
\begin{tabular}{|l|l|}
\hline
\textbf{Kategoria} & \textbf{Przykłady} \\
\hline
Bazy danych & PostgreSQL, MySQL, SQL Server, MongoDB \\
Cache & Redis, Valkey \\
Messaging & RabbitMQ, Azure Service Bus, Kafka \\
Storage & Azure Blob Storage, AWS S3 \\
Cloud Services & Azure, AWS, GCP \\
\hline
\end{tabular}
\end{table}

\subsection{Hosting Integrations vs Client Integrations}

\subsubsection{Hosting Integrations}

\begin{itemize}
    \item Używane w \textbf{App Host}
    \item Zarządzają cyklem życia zasobów
    \item Prefix: \texttt{Aspire.Hosting.*}
    \item Przykład: \texttt{Aspire.Hosting.Redis}
\end{itemize}

\begin{lstlisting}[language={[Sharp]C}]
var redis = builder.AddRedis("cache");
\end{lstlisting}

\subsubsection{Client Integrations}

\begin{itemize}
    \item Używane w \textbf{serwisach aplikacyjnych}
    \item Konfigurują klientów do łączenia się z zasobami
    \item Prefix: \texttt{Aspire.*}
    \item Przykład: \texttt{Aspire. StackExchange.Redis}
\end{itemize}

\begin{lstlisting}[language={[Sharp]C}]
builder.AddRedisClient("cache");
\end{lstlisting}

\subsection{Telemetria i Observability}

\begin{tcolorbox}[colback=red!5!white,colframe=red!75!black,title=OpenTelemetry]
. NET Aspire ma wbudowane wsparcie dla OpenTelemetry, zapewniając kompletną obserwowalność aplikacji. 
\end{tcolorbox}

\subsubsection{Trzy filary observability:}

\begin{enumerate}
    \item \textbf{Logging} -- Strukturalne logi ze wszystkich serwisów
    \item \textbf{Metrics} -- Metryki wydajności i biznesowe
    \item \textbf{Tracing} -- Rozproszone śledzenie requestów przez wszystkie serwisy
\end{enumerate}

\subsubsection{Dashboard}

Aspire dostarcza wbudowany dashboard dostępny podczas developmentu:
\begin{itemize}
    \item Widok wszystkich serwisów i ich statusu
    \item Live logs z wszystkich komponentów
    \item Metryki i wykresy
    \item Distributed tracing
    \item Console output
\end{itemize}

\subsection{Deployment i Manifesty}

\subsubsection{Manifest}

. NET Aspire może wygenerować manifest opisujący całą aplikację:

\begin{lstlisting}[language=bash]
dotnet run --project AppHost --publisher manifest --output-path manifest.json
\end{lstlisting}

\textbf{Manifest zawiera:}
\begin{itemize}
    \item Definicje wszystkich serwisów
    \item Zależności między nimi
    \item Wymagane zasoby (bazy danych, cache, itp.)
    \item Zmienne środowiskowe
    \item Konfigurację portów
\end{itemize}

\subsubsection{Opcje deployment:}

\begin{itemize}
    \item \textbf{Azure Container Apps} -- Natywne wsparcie
    \item \textbf{Kubernetes} -- Przez konwersję manifestu
    \item \textbf{Docker Compose} -- Lokalne środowisko produkcyjne
    \item \textbf{Azure Developer CLI (azd)} -- Szybki deployment do Azure
\end{itemize}

\subsection{Dlaczego używać .NET Aspire?}

\begin{tcolorbox}[colback=orange!5!white,colframe=orange!75!black,title=Korzyści]
\textbf{Zalety używania .NET Aspire:}
\end{tcolorbox}

\begin{enumerate}
    \item \textbf{Uproszczony development lokalny}
    \begin{itemize}
        \item Jeden punkt startowy dla całej aplikacji
        \item Automatyczne zarządzanie zależnościami
        \item Live reload i hot reload
    \end{itemize}
    
    \item \textbf{Production-ready z pudełka}
    \begin{itemize}
        \item Wbudowana telemetria
        \item Health checks
        \item Resilience patterns
    \end{itemize}
    
    \item \textbf{Ujednolicona konfiguracja}
    \begin{itemize}
        \item Service Defaults dla konsystencji
        \item Centralna konfiguracja w App Host
        \item Łatwe zarządzanie sekretami
    \end{itemize}
    
    \item \textbf{Ecosystem integracji}
    \begin{itemize}
        \item Gotowe komponenty dla popularnych technologii
        \item Spójne API dla różnych providerów
        \item Łatwa wymiana implementacji
    \end{itemize}
    
    \item \textbf{Developer Experience}
    \begin{itemize}
        \item Wspaniały dashboard
        \item Intelligent IntelliSense
        \item Visual Studio integration
    \end{itemize}
\end{enumerate}

\subsection{Kiedy używać .NET Aspire?}

\textbf{Idealnie dla:}
\begin{itemize}
    \item Aplikacji mikroservices
    \item Aplikacji cloud-native
    \item Projektów z wieloma serwisami
    \item Zespołów potrzebujących spójnej observability
\end{itemize}

\textbf{Możliwe overkill dla:}
\begin{itemize}
    \item Prostych aplikacji monolitycznych
    \item Single-service projektów
    \item Prototypów bez planów skalowania
\end{itemize}