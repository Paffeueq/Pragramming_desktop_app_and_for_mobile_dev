\documentclass[12pt,a4paper]{article}

% Kodowanie - TYLKO TO:
\usepackage[utf8]{inputenc}
\usepackage[T1]{fontenc}
\usepackage{lmodern}            % WAŻNE dla ł
\usepackage[polish]{babel}

% Reszta pakietów:
\usepackage{hyperref}
\usepackage{graphicx}
\usepackage{listings}
\usepackage{xcolor}
\usepackage{geometry}
\usepackage{fancyhdr}
\usepackage{tcolorbox}

\geometry{margin=2.5cm}

\hypersetup{
	colorlinks=true,
	linkcolor=blue,
	filecolor=magenta,      
	urlcolor=cyan,
	pdftitle={Laboratorium 4 - Blazor . NET 8 - Powtórka},
	pdfauthor={Student},
}

\definecolor{codegreen}{rgb}{0,0.6,0}
\definecolor{codegray}{rgb}{0.5,0.5,0.5}
\definecolor{codepurple}{rgb}{0.58,0,0.82}
\definecolor{backcolour}{rgb}{0.95,0.95,0.92}

\lstdefinestyle{csharpstyle}{
	backgroundcolor=\color{backcolour},   
	commentstyle=\color{codegreen},
	keywordstyle=\color{magenta},
	numberstyle=\tiny\color{codegray},
	stringstyle=\color{codepurple},
	basicstyle=\ttfamily\footnotesize,
	breakatwhitespace=false,         
	breaklines=true,                 
	captionpos=b,                    
	keepspaces=true,                 
	numbers=left,                    
	numbersep=5pt,                  
	showspaces=false,                
	showstringspaces=false,
	showtabs=false,                  
	tabsize=2,
	language=[Sharp]C
}

\lstset{style=csharpstyle}

\pagestyle{fancy}
\fancyhf{}
\rhead{Lab 4 - Blazor .NET 8}
\lhead{Powtórka na kolokwium}
\rfoot{Strona \thepage}

\title{\textbf{Laboratorium 4 - Powtórka} \\ Blazor - Aplikacje w .NET 8}
\author{Materiały na kolokwium}
\date{\today}

\begin{document}
	
	\maketitle
	\tableofcontents
	\newpage
	
	\section{Teoria}

\subsection{Dlaczego Blazor (.NET 8 "Blazor Web App")? }

\begin{tcolorbox}[colback=yellow!10!white,colframe=yellow! 75!black,title=Kluczowe zalety]
\textbf{Jeden szablon = wiele trybów renderowania}

Blazor .NET 8 wprowadza koncepcję \textit{progressive enhancement}, która pozwala na:
\begin{itemize}
    \item Używanie jednego szablonu projektu dla różnych scenariuszy
    \item Elastyczny wybór trybu renderowania na poziomie komponentu
    \item Stopniowe dodawanie interaktywności tam, gdzie jest potrzebna
\end{itemize}
\end{tcolorbox}

\subsubsection{Zalety podejścia}

\begin{enumerate}
    \item \textbf{SEO + szybkość pierwszego ładowania (SSR)} \\
    Server-Side Rendering zapewnia:
    \begin{itemize}
        \item Szybkie wyświetlenie treści (First Contentful Paint)
        \item Indeksowanie przez wyszukiwarki bez JavaScript
        \item Lepsze wskaźniki Core Web Vitals
    \end{itemize}
    
    \item \textbf{Możliwość przejścia do interaktywności} \\
    Nie trzeba przepisywać całej aplikacji - można:
    \begin{itemize}
        \item Zacząć od statycznego SSR
        \item Dodać interaktywność tylko tam, gdzie potrzebna
        \item Mieszać różne tryby w jednej aplikacji
    \end{itemize}
    
    \item \textbf{Wspólny kod .NET} \\
    Model, walidacja, logika biznesowa:
    \begin{itemize}
        \item Dzielone między serwerem a klientem
        \item Jedna baza kodu w C\#
        \item Możliwość reużycia bibliotek . NET
    \end{itemize}
\end{enumerate}

\subsection{Tryby renderowania (@rendermode)}

\begin{tcolorbox}[colback=cyan!10!white,colframe=cyan!75!black,title=Cztery tryby renderowania]
Blazor . NET 8 oferuje cztery główne tryby renderowania, które można mieszać w jednej aplikacji. 
\end{tcolorbox}

\subsubsection{1. Static SSR (brak interaktywności)}

\textbf{Oznaczenie:} brak atrybutu \texttt{@rendermode} lub \texttt{@rendermode="InteractiveServer" @(false)}

\textbf{Charakterystyka:}
\begin{itemize}
    \item Renderowanie tylko po stronie serwera
    \item Brak możliwości obsługi zdarzeń (onClick, onChange, etc.)
    \item Minimalny rozmiar przesyłanych danych
    \item Idealny dla stron statycznych, treści, formularzy bez walidacji
\end{itemize}

\textbf{Przykład:}
\begin{lstlisting}[language={[Sharp]C}]
@page "/about"

<h3>O nas</h3>
<p>Strona statyczna bez interaktywności</p>
\end{lstlisting}

\subsubsection{2. Interactive Server}

\textbf{Oznaczenie:} \texttt{@rendermode="InteractiveServer"}

\textbf{Charakterystyka:}
\begin{itemize}
    \item Logika wykonywana po stronie serwera
    \item Komunikacja przez SignalR/WebSocket
    \item Mały rozmiar początkowego ładowania
    \item Każde zdarzenie wymaga round-trip do serwera
    \item Wymaga stałego połączenia
\end{itemize}

\textbf{Zalety:}
\begin{itemize}
    \item Dostęp do zasobów serwera (baza danych, pliki)
    \item Bezpieczny kod (nie widoczny dla klienta)
    \item Szybkie uruchomienie aplikacji
\end{itemize}

\textbf{Wady:}
\begin{itemize}
    \item Latencja sieci przy każdej interakcji
    \item Wymagane stałe połączenie
    \item Większe obciążenie serwera
    \item Nie działa offline
\end{itemize}

\textbf{Przykład:}
\begin{lstlisting}[language={[Sharp]C}]
@page "/counter"
@rendermode InteractiveServer

<h3>Counter</h3>
<p>Current count: @currentCount</p>
<button @onclick="IncrementCount">Click me</button>

@code {
    private int currentCount = 0;
    
    private void IncrementCount()
    {
        currentCount++;
    }
}
\end{lstlisting}

\subsubsection{3.  Interactive WebAssembly}

\textbf{Oznaczenie:} \texttt{@rendermode="InteractiveWebAssembly"}

\textbf{Charakterystyka:}
\begin{itemize}
    \item Logika wykonywana w przeglądarce
    \item Aplikacja działa po stronie klienta
    \item Większy rozmiar początkowego pobierania
    \item Brak latencji przy interakcjach
    \item Możliwość pracy offline (PWA)
\end{itemize}

\textbf{Zalety:}
\begin{itemize}
    \item Natychmiastowa reakcja na interakcje
    \item Lepsza skalowalność (mniej obciążenia serwera)
    \item Możliwość pracy offline
    \item Możliwość tworzenia PWA
\end{itemize}

\textbf{Wady:}
\begin{itemize}
    \item Większy rozmiar początkowego pobierania (~2-3 MB)
    \item Dłuższy czas pierwszego uruchomienia
    \item Kod widoczny dla klienta
    \item Ograniczony dostęp do zasobów systemowych
\end{itemize}

\textbf{Przykład:}
\begin{lstlisting}[language={[Sharp]C}]
@page "/wasm-counter"
@rendermode InteractiveWebAssembly

<h3>WebAssembly Counter</h3>
<p>Current count: @currentCount</p>
<button @onclick="IncrementCount">Click me</button>

@code {
    private int currentCount = 0;
    
    private void IncrementCount()
    {
        currentCount++;
    }
}
\end{lstlisting}

\subsubsection{4. Interactive Auto}

\textbf{Oznaczenie:} \texttt{@rendermode="InteractiveAuto"}

\textbf{Charakterystyka:}
\begin{itemize}
    \item Pierwszy raz: Interactive Server
    \item W tle pobiera WebAssembly
    \item Kolejne wizyty: Interactive WebAssembly
    \item Best of both worlds
\end{itemize}

\textbf{Zalety:}
\begin{itemize}
    \item Szybkie pierwsze uruchomienie
    \item Późniejsza praca bez latencji
    \item Automatyczne przełączanie
\end{itemize}

\textbf{Wady:}
\begin{itemize}
    \item Większa złożoność
    \item Trudniejszy debugging
    \item Kod musi być kompatybilny z oboma trybami
\end{itemize}

\subsubsection{Wybór trybu renderowania}

Tryb można wybrać na trzech poziomach:

\begin{enumerate}
    \item \textbf{Globalnie} - w \texttt{App.razor}:
\begin{lstlisting}[language={[Sharp]C}]
<Routes @rendermode="InteractiveServer" />
\end{lstlisting}

    \item \textbf{Per strona} - w komponencie strony:
\begin{lstlisting}[language={[Sharp]C}]
@page "/mypage"
@rendermode InteractiveServer
\end{lstlisting}

    \item \textbf{Per komponent} - przy użyciu komponentu:
\begin{lstlisting}[language={[Sharp]C}]
<MyComponent @rendermode="InteractiveServer" />
\end{lstlisting}
\end{enumerate}

\subsection{Koszt i trade-offy}

\begin{table}[h]
\centering
\begin{tabular}{|l|l|l|}
\hline
\textbf{Tryb} & \textbf{Zalety (+)} & \textbf{Wady (-)} \\
\hline
\textbf{Static SSR} & 
\begin{tabular}[t]{@{}l@{}}
Minimalny rozmiar \\
Szybkie ładowanie \\
Dobre SEO
\end{tabular} & 
\begin{tabular}[t]{@{}l@{}}
Brak interaktywności \\
Tylko proste formularze
\end{tabular} \\
\hline
\textbf{Server} & 
\begin{tabular}[t]{@{}l@{}}
Mały rozmiar startowy \\
Dostęp do serwera \\
Bezpieczny kod
\end{tabular} & 
\begin{tabular}[t]{@{}l@{}}
Stałe połączenia \\
Latency roundtrip \\
Nie działa offline
\end{tabular} \\
\hline
\textbf{WebAssembly} & 
\begin{tabular}[t]{@{}l@{}}
Skalowalność \\
Praca offline \\
PWA możliwe \\
Brak latencji
\end{tabular} & 
\begin{tabular}[t]{@{}l@{}}
Większy download \\
Dłuższe uruchomienie \\
Kod widoczny
\end{tabular} \\
\hline
\textbf{Auto} & 
\begin{tabular}[t]{@{}l@{}}
Szybki start \\
Potem bez latencji \\
Lepsze UX
\end{tabular} & 
\begin{tabular}[t]{@{}l@{}}
Większa złożoność \\
Trudniejszy debug
\end{tabular} \\
\hline
\end{tabular}
\caption{Porównanie trybów renderowania}
\end{table}

\subsection{Stream Rendering \& Enhanced Navigation}

\subsubsection{Stream Rendering}

\textbf{Stream Rendering} pozwala na wyświetlenie części strony natychmiast, a następnie zaktualizowanie jej po załadowaniu wolnych danych.

\textbf{Przykład:}
\begin{lstlisting}[language={[Sharp]C}]
@attribute [StreamRendering(true)]

@if (forecasts == null)
{
    <p>Loading...</p>
}
else
{
    <table>
        @foreach (var forecast in forecasts)
        {
            <tr><td>@forecast.Date</td></tr>
        }
    </table>
}

@code {
    private WeatherForecast[]? forecasts;
    
    protected override async Task OnInitializedAsync()
    {
        // Symulacja wolnego zapytania
        await Task.Delay(2000);
        forecasts = await GetForecasts();
    }
}
\end{lstlisting}

\textbf{Działanie:}
\begin{enumerate}
    \item Serwer natychmiast zwraca HTML z "Loading..."
    \item W tle wykonuje się \texttt{OnInitializedAsync()}
    \item Po zakończeniu serwer streamuje zaktualizowany HTML
    \item Przeglądarka podmienia zawartość
\end{enumerate}

\subsubsection{Enhanced Navigation}

\textbf{Enhanced Navigation} przyspiesza nawigację między stronami statycznymi SSR poprzez:
\begin{itemize}
    \item Pobieranie tylko zmienionych części strony (nie cały dokument)
    \item Zachowanie JavaScript state
    \item Szybsze przejścia (podobne do SPA)
\end{itemize}

Włączane domyślnie w \texttt{App.razor}:
\begin{lstlisting}[language={[Sharp]C}]
<head>
    <HeadOutlet @rendermode="InteractiveServer" />
</head>
\end{lstlisting}

\subsection{Struktura projektu ``Blazor Web App''}  % Use LaTeX quotes instead


Typowy projekt Blazor Web App (. NET 8) ma następującą strukturę:

\begin{verbatim}
MyBlazorApp/
├── MyBlazorApp/              # Projekt główny (Server)
│   ├── Components/
│   │   ├── Layout/
│   │   ├── Pages/
│   │   └── App.razor
│   ├── wwwroot/
│   ├── appsettings.json
│   └── Program.cs
│
└── MyBlazorApp.Client/       # Projekt WebAssembly
    ├── Pages/
    ├── wwwroot/
    └── Program.cs
\end{verbatim}

\textbf{Wyjaśnienie:}
\begin{itemize}
    \item \textbf{MyBlazorApp} - projekt główny, renderuje SSR i Server
    \item \textbf{MyBlazorApp.Client} - projekt dla WebAssembly
    \item Komponenty mogą być współdzielone przez oba projekty
\end{itemize}

\subsection{Formularze w Blazor}

\subsubsection{EditForm}

Podstawowy komponent do tworzenia formularzy z walidacją:

\begin{lstlisting}[language={[Sharp]C}]
<EditForm Model="@model" OnValidSubmit="HandleValidSubmit">
    <DataAnnotationsValidator />
    <ValidationSummary />
    
    <div>
        <label>Nazwa:</label>
        <InputText @bind-Value="model. Name" />
        <ValidationMessage For="@(() => model.Name)" />
    </div>
    
    <button type="submit">Zapisz</button>
</EditForm>

@code {
    private PersonModel model = new();
    
    private void HandleValidSubmit()
    {
        // Zapisz dane
    }
}
\end{lstlisting}

\subsubsection{Model z walidacją}

\begin{lstlisting}[language={[Sharp]C}]
using System

\end{lstlisting}
	\newpage
	\section{Materiały}

\subsection{Filmy instruktażowe}

\begin{tcolorbox}[colback=blue!5!white,colframe=blue!75!black,title=Materiały wideo]
\begin{itemize}
    \item \href{https://www.youtube.com/watch?v=YqJgIlUWwM8}{Blazor SSR vs Server vs WebAssembly vs Auto} \\
    Omówienie różnic między trybami renderowania w Blazor . NET 8
    
    \item \href{https://www.youtube.com/watch? v=eUulbbY30B4}{CRUD z wszystkimi trybami renderowania (. NET 8)} \\
    Praktyczne przykłady implementacji operacji CRUD
    
    \item \href{https://www. youtube.com/watch?v=Uby6TUPIBbs}{CRUD z trybem WebAssembly (.NET 8)} \\
    Szczegółowa implementacja w trybie WebAssembly
\end{itemize}
\end{tcolorbox}

\subsection{Dokumentacja Microsoft}

\begin{tcolorbox}[colback=green!5!white,colframe=green!75!black,title=Dokumentacja oficjalna]
\begin{itemize}
    \item \href{https://learn.microsoft.com/en-us/aspnet/core/blazor/}{Blazor (ASP.NET Core)} \\
    Główna dokumentacja frameworka Blazor
    
    \item \href{https://learn.microsoft.com/en-us/aspnet/core/blazor/components/render-modes}{Render modes (.NET 8)} \\
    Szczegółowy opis trybów renderowania
    
    \item \href{https://learn.microsoft.com/en-us/aspnet/core/blazor/components/}{Komponenty i routing} \\
    Tworzenie i zarządzanie komponentami
    
    \item \href{https://learn.microsoft.com/en-us/aspnet/core/blazor/forms/}{Forms \& EditForm} \\
    Praca z formularzami w Blazor
    
    \item \href{https://learn. microsoft.com/en-us/aspnet/core/blazor/components/rendering}{Stream rendering / Enhanced navigation} \\
    Zaawansowane techniki renderowania
    
    \item \href{https://learn. microsoft.com/en-us/aspnet/core/blazor/tutorials/signalr-blazor}{SignalR (Blazor Server)} \\
    Komunikacja w czasie rzeczywistym
    
    \item \href{https://learn.microsoft.com/en-us/ef/core/}{Entity Framework Core} \\
    ORM do pracy z bazami danych
    
    \item \href{https://learn.microsoft.com/en-us/aspnet/core/blazor/host-and-deploy/}{Hosting i publikacja} \\
    Wdrażanie aplikacji Blazor
    
    \item \href{https://learn.microsoft.com/en-us/aspnet/core/blazor/progressive-web-app}{PWA (Blazor WebAssembly)} \\
    Tworzenie Progressive Web Apps
\end{itemize}
\end{tcolorbox}

\subsection{Repozytoria przykładowe}

\begin{tcolorbox}[colback=orange!5!white,colframe=orange!75!black,title=Przykładowy kod źródłowy]
\begin{itemize}
    \item \href{https://github.com/dotnet/blazor-samples}{GitHub: dotnet/blazor-samples} \\
    Oficjalne przykłady od Microsoft
    
    \item \href{https://github.com/aspnetcore/AspNetCore.Docs. Samples/tree/main/blazor}{GitHub: AspNetCore.Docs.Samples} \\
    Przykłady z dokumentacji
\end{itemize}
\end{tcolorbox}

\subsection{Dodatkowe zasoby}

\begin{itemize}
    \item \href{https://dotnet.microsoft.com/apps/aspnet/web-apps/blazor}{Strona główna Blazor}
    \item \href{https://blazor.net/}{Blazor University} - nieoficjalny przewodnik
    \item \href{https://www.blazorcomponents.com/}{Biblioteka komponentów Blazor}
\end{itemize}
	\newpage
	\section{Cheat Sheet - Ściąga na kolokwium}

\subsection{Tryby renderowania - składnia}

\begin{tcolorbox}[colback=red!10!white,colframe=red!75!black,title=NAJWAŻNIEJSZE! ]
\begin{lstlisting}
// 1. Brak interaktywności
@page "/static"
<h3>Static SSR</h3>

// 2. Interactive Server
@page "/server"
@rendermode InteractiveServer

// 3. Interactive WebAssembly
@page "/wasm"
@rendermode InteractiveWebAssembly

// 4. Interactive Auto
@page "/auto"
@rendermode InteractiveAuto
\end{lstlisting}
\end{tcolorbox}

\subsection{Stream Rendering}

\begin{lstlisting}
@attribute [StreamRendering(true)]

@if (data == null)
{
    <p>Loading...</p>
}
else
{
    <!-- Wyświetl dane -->
}

@code {
    private Data?  data;
    
    protected override async Task OnInitializedAsync()
    {
        data = await FetchDataAsync();
    }
}
\end{lstlisting}

\subsection{Formularze}

\begin{lstlisting}
<EditForm Model="@model" OnValidSubmit="HandleSubmit">
    <DataAnnotationsValidator />
    <ValidationSummary />
    
    <InputText @bind-Value="model. Name" />
    <ValidationMessage For="@(() => model. Name)" />
    
    <button type="submit">Zapisz</button>
</EditForm>

@code {
    private MyModel model = new();
    
    private void HandleSubmit()
    {
        // Zapisz
    }
}
\end{lstlisting}

\subsection{Walidacja modelu}

\begin{lstlisting}
using System.ComponentModel.DataAnnotations;

public class MyModel
{
    [Required(ErrorMessage = "Pole wymagane")]
    public string Name { get; set; } = "";
    
    [Range(1, 100)]
    public int Age { get; set; }
    
    [EmailAddress]
    public string Email { get; set; } = "";
    
    [Compare("Password")]
    public string ConfirmPassword { get; set; } = "";
}
\end{lstlisting}

\subsection{Entity Framework}

\begin{lstlisting}
// DbContext
public class AppDbContext : DbContext
{
    public AppDbContext(DbContextOptions<AppDbContext> options)
        : base(options) { }
    
    public DbSet<Entity> Entities { get; set; }
}

// Program. cs
builder.Services.AddDbContext<AppDbContext>(options =>
    options.UseSqlite("Data Source=app.db"));

// Użycie
@inject AppDbContext Db

@code {
    private List<Entity> items = new();
    
    protected override async Task OnInitializedAsync()
    {
        items = await Db.Entities.ToListAsync();
    }
    
    private async Task Add(Entity entity)
    {
        Db.Entities.Add(entity);
        await Db.SaveChangesAsync();
    }
    
    private async Task Delete(int id)
    {
        var item = await Db.Entities. FindAsync(id);
        if (item != null)
        {
            Db.Entities. Remove(item);
            await Db.SaveChangesAsync();
        }
    }
}
\end{lstlisting}

\subsection{Dependency Injection}

\begin{lstlisting}
// Program.cs - rejestracja
builder.Services.AddScoped<IMyService, MyService>();
builder.Services.AddSingleton<ICacheService, CacheService>();
builder.Services.AddTransient<IEmailService, EmailService>();

// Komponent - użycie
@inject IMyService MyService

@code {
    protected override async Task OnInitializedAsync()
    {
        var data = await MyService.GetDataAsync();
    }
}
\end{lstlisting}

\subsection{HTTP w WebAssembly}

\begin{lstlisting}
@inject HttpClient Http

@code {
    // GET
    var items = await Http.GetFromJsonAsync<List<Item>>(
        "api/items");
    
    // POST
    await Http.PostAsJsonAsync("api/items", newItem);
    
    // PUT
    await Http.PutAsJsonAsync($"api/items/{id}", item);
    
    // DELETE
    await Http.DeleteAsync($"api/items/{id}");
}
\end{lstlisting}

\subsection{Nawigacja}

\begin{lstlisting}
@inject NavigationManager Navigation

<button @onclick="Navigate">Przejdź</button>

@code {
    private void Navigate()
    {
        Navigation.NavigateTo("/other-page");
    }
}
\end{lstlisting}

\subsection{Parametry komponentu}

\begin{lstlisting}
// Definicja
@code {
    [Parameter]
    public string Title { get; set; } = "";
    
    [Parameter]
    public EventCallback OnClick { get; set; }
}

// Użycie
<MyComponent Title="Hello" OnClick="HandleClick" />
\end{lstlisting}

\subsection{Lifecycle hooks}

\begin{lstlisting}
protected override void OnInitialized()
{
    // Synchroniczny, pierwszy
}

protected override async Task OnInitializedAsync()
{
    // Asynchroniczny, do API/DB
}

protected override void OnParametersSet()
{
    // Po każdej zmianie parametrów
}

protected override async Task OnAfterRenderAsync(bool firstRender)
{
    if (firstRender)
    {
        // JS interop, focus, etc.
    }
}

public void Dispose()
{
    // Cleanup
}
\end{lstlisting}

\subsection{Binding}

\begin{lstlisting}
// One-way
<p>@text</p>

// Two-way
<input @bind="text" />
<input @bind="text" @bind:event="oninput" />

// Checkbox
<input type="checkbox" @bind="isChecked" />

// Select
<select @bind="selectedValue">
    <option value="1">Opcja 1</option>
    <option value="2">Opcja 2</option>
</select>
\end{lstlisting}

\subsection{Conditional rendering}

\begin{lstlisting}
@if (condition)
{
    <p>True</p>
}
else
{
    <p>False</p>
}

@switch (value)
{
    case 1:
        <p>Jeden</p>
        break;
    case 2:
        <p>Dwa</p>
        break;
    default:
        <p>Inne</p>
        break;
}
\end{lstlisting}

\subsection{Loops}

\begin{lstlisting}
@foreach (var item in items)
{
    <div>@item.Name</div>
}

@for (int i = 0; i < 10; i++)
{
    <p>@i</p>
}
\end{lstlisting}

\subsection{Event handling}

\begin{lstlisting}
<button @onclick="HandleClick">Click</button>
<input @onchange="HandleChange" />
<input @oninput="HandleInput" />
<form @onsubmit="HandleSubmit">

@code {
    private void HandleClick()
    {
        // ... 
    }
    
    private void HandleClick(MouseEventArgs e)
    {
        // Z argumentami
    }
    
    private async Task HandleClickAsync()
    {
        // Asynchroniczny
    }
}
\end{lstlisting}

\subsection{Kiedy którego trybu użyć - tabela decyzyjna}

\begin{table}[h]
\centering
\small
\begin{tabular}{|l|l|}
\hline
\textbf{Scenariusz} & \textbf{Tryb} \\
\hline
Strona "O nas", blog & Static SSR \\
\hline
Dashboard firmowy & Interactive Server \\
\hline
Edytor graficzny & Interactive WebAssembly \\
\hline
PWA, praca offline & Interactive WebAssembly \\
\hline
Aplikacja SaaS & Interactive Auto \\
\hline
Prototyp & Interactive Server \\
\hline
Wysoki ruch, publiczna & Interactive WebAssembly \\
\hline
Wymaga DB access & Interactive Server \\
\hline
\end{tabular}
\caption{Decyzja o trybie renderowania}
\end{table}

\subsection{Najczęstsze błędy}

\begin{tcolorbox}[colback=red!10!white,colframe=red!75!black,title=UWAGA!]
\textbf{1. Brak @rendermode przy interaktywności}
\begin{lstlisting}
// ŹLE - @onclick nie zadziała
@page "/counter"
<button @onclick="Increment">Count</button>

// DOBRZE
@page "/counter"
@rendermode InteractiveServer
<button @onclick="Increment">Count</button>
\end{lstlisting}

\textbf{2. Używanie DbContext w WebAssembly}
\begin{lstlisting}
// ŹLE - WebAssembly nie ma dostępu do DB
@rendermode InteractiveWebAssembly
@inject AppDbContext Db

// DOBRZE - użyj HTTP API
@rendermode InteractiveWebAssembly
@inject HttpClient Http
\end{lstlisting}

\textbf{3. Zapomnienie await}
\begin{lstlisting}
// ŹLE
var data = Db.Items.ToListAsync();

// DOBRZE
var data = await Db. Items.ToListAsync();
\end{lstlisting}

\textbf{4. Brak DataAnnotationsValidator}
\begin{lstlisting}
// ŹLE - walidacja nie działa
<EditForm Model="@model">
    <InputText @bind-Value="model. Name" />
</EditForm>

// DOBRZE
<EditForm Model="@model">
    <DataAnnotationsValidator />
    <InputText @bind-Value="model. Name" />
</EditForm>
\end{lstlisting}
\end{tcolorbox}

\subsection{Polecenia CLI}

\begin{tcolorbox}[colback=green! 10!white,colframe=green!75!black,title=Przydatne komendy]
\begin{verbatim}
# Nowy projekt
dotnet new blazor -o MyApp -int Server

# Dodanie pakietu
dotnet add package Microsoft.EntityFrameworkCore. Sqlite

# Migracje EF Core
dotnet ef migrations add InitialCreate
dotnet ef database update

# Uruchomienie
dotnet run

# Publikacja
dotnet publish -c Release
\end{verbatim}
\end{tcolorbox}

	
\end{document}