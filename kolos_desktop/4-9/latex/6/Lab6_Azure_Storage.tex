\documentclass[a4paper,11pt]{article}
\usepackage[utf8]{inputenc}
\usepackage[polish]{babel}
\usepackage[T1]{fontenc}
\usepackage{geometry}
\usepackage{xcolor}
\usepackage{enumitem}
\usepackage{graphicx}
\usepackage{tcolorbox}
\usepackage{hyperref}
\usepackage{amsmath}
\usepackage{amssymb}

\geometry{left=2cm, right=2cm, top=2cm, bottom=2cm}

\definecolor{azureblue}{RGB}{0,120,215}
\definecolor{lightblue}{RGB}{230,240,250}
\definecolor{darkblue}{RGB}{0,90,158}

\newtcolorbox{infobox}[1]{
    colback=lightblue,
    colframe=azureblue,
    fonttitle=\bfseries,
    title=#1
}

\newtcolorbox{warnbox}{
    colback=yellow! 10,
    colframe=orange,
    fonttitle=\bfseries,
    title=Ważne!
}

\title{\textbf{\Huge Lab 6 - Azure Storage} \\ \Large Blob, Queue, Table Storage}
\author{Materiały na kolokwium - Część 2}
\date{\today}

\begin{document}

\maketitle
\tableofcontents
\newpage

\section{Azure Storage Account}

\subsection{Podstawy}

\begin{infobox}{Co to jest Storage Account?}
\textbf{Storage Account} to namespace dla wszystkich usług storage w Azure. 

\textbf{Zawiera 4 usługi:}
\begin{itemize}
    \item \textbf{Blob Storage} - object storage (pliki binarne)
    \item \textbf{Queue Storage} - message queue
    \item \textbf{Table Storage} - NoSQL key-value store
    \item \textbf{File Storage} - SMB file shares (network drives)
\end{itemize}

\textbf{Unikalna nazwa:}
\begin{itemize}
    \item Globalnie unikalna w całym Azure
    \item 3-24 znaki
    \item Tylko małe litery i cyfry
    \item Przykład: \texttt{mystorageaccount123}
\end{itemize}

\textbf{Endpoints:}
\begin{itemize}
    \item Blob: \texttt{https://\{account\}.blob.core.windows.net}
    \item Queue: \texttt{https://\{account\}.queue.core.windows.net}
    \item Table: \texttt{https://\{account\}.table.core.windows. net}
    \item File: \texttt{https://\{account\}.file.core.windows.net}
\end{itemize}
\end{infobox}

\subsection{Typy Kont Storage}

\begin{infobox}{General-purpose v2 (GPv2)}
\textbf{Charakterystyka:}
\begin{itemize}
    \item Rekomendowany typ (current standard)
    \item Obsługuje wszystkie usługi storage
    \item Obsługuje wszystkie access tiers (Hot, Cool, Cold, Archive)
    \item Lowest prices
    \item Best dla większości scenariuszy
\end{itemize}

\textbf{Use case:} Default choice dla nowych projektów
\end{infobox}

\begin{infobox}{Premium Storage}
\textbf{Charakterystyka:}
\begin{itemize}
    \item SSD-based storage
    \item Wyższa wydajność (niższa latencja)
    \item Wyższe IOPS
    \item Droższe
\end{itemize}

\textbf{3 typy Premium:}
\begin{itemize}
    \item \textbf{Premium Block Blobs} - dla block i append blobs
    \item \textbf{Premium File Shares} - dla Azure Files
    \item \textbf{Premium Page Blobs} - dla page blobs (VHD)
\end{itemize}

\textbf{Use case:}
\begin{itemize}
    \item High-transaction workloads
    \item Low-latency requirements
    \item Database files
    \item VMs (Premium Page Blobs)
\end{itemize}
\end{infobox}

\subsection{Replication (Redundancja)}

\begin{infobox}{LRS - Locally Redundant Storage}
\textbf{Charakterystyka:}
\begin{itemize}
    \item 3 synchroniczne kopie w jednym datacenter
    \item Najtańsza opcja
    \item 11 nines durability (99.999999999\% w roku)
\end{itemize}

\textbf{Ochrona przed:}
\begin{itemize}
    \item Disk failure
    \item  Rack failure
    \item  Hardware problems
\end{itemize}

\textbf{NIE chroni przed:}
\begin{itemize}
    \item  Datacenter failure (fire, flood, earthquake)
    \item Regional disaster
\end{itemize}

\textbf{Use case:}
\begin{itemize}
    \item Dev/test environments
    \item Dane które można odtworzyć
    \item Gdy nie potrzebujesz high availability
    \item Optymalizacja kosztów
\end{itemize}
\end{infobox}

\begin{infobox}{ZRS - Zone Redundant Storage}
\textbf{Charakterystyka:}
\begin{itemize}
    \item 3 synchroniczne kopie w 3 różnych availability zones
    \item 12 nines durability (99. 9999999999\%)
    \item Wyższy koszt niż LRS
\end{itemize}

\textbf{Availability Zones:}
\begin{itemize}
    \item Fizycznie oddzielone datacenter w tym samym regionie
    \item Niezależne zasilanie, chłodzenie, networking
    \item Odległość: kilometry od siebie
\end{itemize}

\textbf{Ochrona przed:}
\begin{itemize}
    \item  Wszystko co LRS
    \item Datacenter failure w regionie
\end{itemize}

\textbf{NIE chroni przed:}
\begin{itemize}
    \item Regional disaster (cały region down)
\end{itemize}

\textbf{Use case:}
\begin{itemize}
    \item Production workloads
    \item High availability w regionie
    \item Gdy potrzebujesz consistency i wysoką dostępność
\end{itemize}
\end{infobox}

\begin{infobox}{GRS - Geo Redundant Storage}
\textbf{Charakterystyka:}
\begin{itemize}
    \item LRS w primary region + asynchroniczna kopia do secondary region
    \item 16 nines durability (99. 99999999999999\%)
    \item Secondary region: setki kilometrów od primary
    \item Paired regions (z góry określone przez Azure)
\end{itemize}

\textbf{Asynchronous replication:}
\begin{itemize}
    \item Opóźnienie: zwykle < 15 minut
    \item RPO (Recovery Point Objective): ~15 min data loss możliwy
    \item Automatic replication
\end{itemize}

\textbf{Secondary region access:}
\begin{itemize}
    \item Default: read-only (nie możesz czytać)
    \item Po failover: becomes primary (read-write)
    \item RA-GRS: Read-Access GRS (możesz czytać ze secondary)
\end{itemize}

\textbf{Ochrona przed:}
\begin{itemize}
    \item  Wszystko co LRS
    \item Regional disaster
    \item  Catastrophic failure całego regionu
\end{itemize}

\textbf{Use case:}
\begin{itemize}
    \item Business-critical data
    \item Disaster recovery requirements
    \item Compliance (data residency in multiple regions)
\end{itemize}
\end{infobox}

\begin{infobox}{GZRS - Geo-Zone Redundant Storage}
\textbf{Charakterystyka:}
\begin{itemize}
    \item ZRS w primary region + GRS (async copy do secondary)
    \item Najwyższa dostępność i durability
    \item 16 nines durability
    \item Najdroższe
\end{itemize}

\textbf{Kombinacja:}
\begin{itemize}
    \item ZRS protection w primary region
    \item GRS protection dla disaster recovery
    \item Best of both worlds
\end{itemize}

\textbf{Use case:}
\begin{itemize}
    \item Maximum consistency, availability, durability
    \item Mission-critical applications
    \item Financial services, healthcare
    \item Gdy nie możesz stracić danych
\end{itemize}
\end{infobox}

\begin{warnbox}
\textbf{Paired Regions:}

Azure ma z góry określone pary regionów dla GRS:
\begin{itemize}
    \item West Europe  North Europe
    \item East US  West US
    \item Southeast Asia  East Asia
\end{itemize}

NIE możesz wybrać custom secondary region! 

\textbf{Failover:}
\begin{itemize}
    \item Automatic failover: NIE (musisz ręcznie zainicjować)
    \item Manual failover: TAK (w Azure Portal)
    \item Czas: ~1 godzina (RTO - Recovery Time Objective)
\end{itemize}
\end{warnbox}

\section{Blob Storage}

\subsection{Podstawy}

\begin{infobox}{Blob Storage - Object Storage}
\textbf{Definicja:} Massively scalable object storage dla unstructured data

\textbf{Unstructured data:}
\begin{itemize}
    \item Dane bez określonego formatu/schematu
    \item Binary data
    \item Przykłady: obrazy, wideo, audio, dokumenty, logi
\end{itemize}

\textbf{Zastosowanie:}
\begin{itemize}
    \item Serving images/documents do przeglądarki
    \item Storing files for distributed access
    \item Streaming video i audio
    \item Backup i disaster recovery
    \item Archiwizacja danych
    \item Data for analysis (big data)
    \item Static website hosting
\end{itemize}

\textbf{Skala:}
\begin{itemize}
    \item Exabytes of data (praktycznie unlimited)
    \item Miliony żądań na sekundę
    \item Global distribution
\end{itemize}
\end{infobox}

\subsection{Hierarchia Blob Storage}

\begin{infobox}{Hierarchia}
\textbf{3-poziomowa struktura:}
\begin{center}
\textbf{Storage Account} → \textbf{Container} → \textbf{Blob}
\end{center}

\textbf{Storage Account:}
\begin{itemize}
    \item Top-level namespace
    \item Unikalna nazwa globalnie
    \item Authentication i authorization level
\end{itemize}

\textbf{Container:}
\begin{itemize}
    \item Logiczne grupowanie blobów (jak folder, ale flat)
    \item Nazwa: 3-63 znaki, lowercase, cyfry, dash
    \item Public access level (Private / Blob / Container)
    \item Unlimited liczba blobów
    \item Unlimited liczba containerów w account
\end{itemize}

\textbf{Blob:}
\begin{itemize}
    \item Actual file/object
    \item Max size zależy od typu (do 4. 75 TB dla block blobs)
    \item Unique URL: \texttt{https://\{account\}.blob.core.windows.net/\{container\}/\{blob\}}
\end{itemize}

\textbf{Virtual directories:}
\begin{itemize}
    \item Blob Storage jest FLAT (nie ma prawdziwych folderów)
    \item Możesz używać "/" w nazwie blobu: \texttt{folder1/folder2/file.txt}
    \item Azure Portal pokazuje to jako foldery (wirtualne)
\end{itemize}
\end{infobox}

\subsection{Typy Blobów}

\begin{infobox}{Block Blobs}
\textbf{Charakterystyka:}
\begin{itemize}
    \item Najbardziej popularny typ
    \item Składa się z bloków (chunks)
    \item Max 50,000 bloków
    \item Max rozmiar bloku: 4000 MiB
    \item Max total size: ~4.75 TB
\end{itemize}

\textbf{Uploading:}
\begin{itemize}
    \item Można uploadować bloki równolegle (paralelizacja)
    \item Po upload wszystkich bloków → commit block list
    \item Atomic operation (albo wszystkie bloki albo żadne)
    \item Resumable uploads (jeśli fail, można wznowić)
\end{itemize}

\textbf{Optimized for:}
\begin{itemize}
    \item Uploading large files efficiently
    \item Streaming scenarios
    \item Sequential reading
\end{itemize}

\textbf{Use case:}
\begin{itemize}
    \item Obrazy, wideo, audio
    \item Dokumenty (PDF, Word, Excel)
    \item Backups
    \item Application data files
    \item 95\% przypadków użycia
\end{itemize}
\end{infobox}

\begin{infobox}{Append Blobs}
\textbf{Charakterystyka:}
\begin{itemize}
    \item Zoptymalizowane pod operacje append (dopisywanie)
    \item Składa się z bloków (append blocks)
    \item Max 50,000 bloków
    \item Max rozmiar bloku: 4 MiB
    \item Max total size: ~195 GB
\end{itemize}

\textbf{Operations:}
\begin{itemize}
    \item Append - TAK (add data at the end)
    \item Update existing blocks - NIE
    \item Delete blocks - NIE
    \item Read - TAK
\end{itemize}

\textbf{Optimized for:}
\begin{itemize}
    \item Append-only scenarios
    \item Data nie jest modyfikowany po zapisie
\end{itemize}

\textbf{Use case:}
\begin{itemize}
    \item Application logs (ciągłe dopisywanie)
    \item Audit logs
    \item Streaming data capture
    \item Event data
    \item Telemetry data
\end{itemize}
\end{infobox}

\begin{infobox}{Page Blobs}
\textbf{Charakterystyka:}
\begin{itemize}
    \item Collection of 512-byte pages
    \item Zoptymalizowane pod random read/write
    \item Max size: 8 TB
    \item Efficient random access
\end{itemize}

\textbf{Operations:}
\begin{itemize}
    \item Read/write arbitrary 512-byte pages
    \item Random access patterns
    \item Low latency
\end{itemize}

\textbf{Optimized for:}
\begin{itemize}
    \item Frequent random read/write operations
    \item Low-latency access
\end{itemize}

\textbf{Use case:}
\begin{itemize}
    \item Virtual Hard Disks (VHD) dla Azure VMs
    \item Database files
    \item Random access data
    \item Azure Premium Storage disks
\end{itemize}

\begin{warnbox}
Page Blobs są głównie używane wewnętrznie przez Azure dla VM disks. 

Rzadko używane bezpośrednio przez developerów. 
\end{warnbox}
\end{infobox}

\subsection{Access Tiers}

\begin{infobox}{Hot Tier - Częsty Dostęp}
\textbf{Charakterystyka:}
\begin{itemize}
    \item Dla aktywnie używanych danych
    \item Najwyższe storage cost
    \item Najniższe access cost (read/write)
    \item Default tier dla nowych blobów
    \item Lowest latency
\end{itemize}

\textbf{Pricing:}
\begin{itemize}
    \item Storage: najdroższe (\$\$/GB/month)
    \item Access: najtańsze (operacje read/write)
\end{itemize}

\textbf{Use case:}
\begin{itemize}
    \item Active data w użyciu
    \item Dane często odczytywane/modyfikowane
    \item Staging area dla ETL
    \item Production databases
    \item Aplikacje wymagające low latency
\end{itemize}

\textbf{Kiedy używać:}
\begin{itemize}
    \item Dostęp do danych > 1 raz dziennie
    \item Short-term storage
\end{itemize}
\end{infobox}

\begin{infobox}{Cool Tier - Rzadki Dostęp}
\textbf{Charakterystyka:}
\begin{itemize}
    \item Dla rzadko używanych danych
    \item Niższe storage cost niż Hot
    \item Wyższe access cost niż Hot
    \item Minimum storage duration: 30 dni
    \item Early deletion fee (jeśli < 30 dni)
\end{itemize}

\textbf{Pricing trade-off:}
\begin{itemize}
    \item Płacisz mniej za storage
    \item Płacisz więcej za każdy access (read/write/list)
\end{itemize}

\textbf{Use case:}
\begin{itemize}
    \item Short-term backup
    \item Older data rarely accessed
    \item Disaster recovery datasets
    \item Media content (stare wideo)
    \item Data analytics datasets (after processing)
\end{itemize}

\textbf{Kiedy używać:}
\begin{itemize}
    \item Dostęp do danych ~ 1 raz na miesiąc
    \item Storage period > 30 dni
\end{itemize}

\begin{warnbox}
Early deletion charge: jeśli usuniesz blob przed 30 dniami, płacisz jakbyś go trzymał przez pełne 30 dni! 
\end{warnbox}
\end{infobox}

\begin{infobox}{Cold Tier - Bardzo Rzadki Dostęp}
\textbf{Charakterystyka:}
\begin{itemize}
    \item Dla bardzo rzadko używanych danych
    \item Jeszcze niższe storage cost
    \item Jeszcze wyższe access cost
    \item Minimum storage duration: 90 dni
    \item Early deletion fee (jeśli < 90 dni)
\end{itemize}

\textbf{Pricing:}
\begin{itemize}
    \item Storage: tańsze niż Cool
    \item Access: droższe niż Cool
\end{itemize}

\textbf{Use case:}
\begin{itemize}
    \item Long-term backup
    \item Compliance data (must keep)
    \item Archival data z occasional access
\end{itemize}

\textbf{Kiedy używać:}
\begin{itemize}
    \item Dostęp ~ raz na kwartał lub rzadziej
    \item Storage period > 90 dni
\end{itemize}
\end{infobox}

\begin{infobox}{Archive Tier - Archiwum}
\textbf{Charakterystyka:}
\begin{itemize}
    \item Offline storage (dane niedostępne bezpośrednio)
    \item Najniższe storage cost
    \item Najwyższe access cost + rehydration cost
    \item Minimum storage duration: 180 dni
    \item Early deletion fee (jeśli < 180 dni)
    \item Rehydration required przed odczytem
\end{itemize}

\textbf{Rehydration:}
\begin{itemize}
    \item Proces przenoszenia blobu z Archive do Hot/Cool
    \item Standard priority: do 15 godzin
    \item High priority: < 1 godzina (dla blobów < 10 GB)
    \item Koszt rehydration + destination tier storage
\end{itemize}

\textbf{2 metody rehydration:}
\begin{enumerate}
    \item \textbf{Copy to Hot/Cool} - oryginalny zostaje w Archive
    \item \textbf{Change tier} - blob przechodzi do Hot/Cool
\end{enumerate}

\textbf{Use case:}
\begin{itemize}
    \item Long-term archival (lata)
    \item Compliance data (legal requirements)
    \item Tape replacement
    \item Backups rarely/never restored
    \item Historical data
\end{itemize}

\textbf{Kiedy używać:}
\begin{itemize}
    \item Dostęp bardzo rzadki (raz na rok lub nigdy)
    \item Storage period > 180 dni
    \item Można czekać godziny na dostęp
\end{itemize}

\begin{warnbox}
Archive = OFFLINE storage! 

NIE możesz bezpośrednio czytać/modyfikować archived blobs. 

Musisz najpierw REHYDRATE (kopiować do Hot/Cool). 
\end{warnbox}
\end{infobox}

\subsection{Zmiana Access Tier}

\begin{infobox}{Tier Changes}
\textbf{Lifecycle Management:}
\begin{itemize}
    \item Możesz ustawić automatyczne reguły
    \item Przykład: po 30 dniach → Cool, po 90 dniach → Archive
    \item Bazuje na last modified time lub creation time
    \item Filters: prefix, blob type
\end{itemize}

\textbf{Manual tier change:}
\begin{itemize}
    \item Możesz zmienić tier ręcznie (Portal, API, CLI)
    \item Hot  Cool  Cold: natychmiastowe
    \item → Archive: natychmiastowe
    \item Archive →: wymaga rehydration
\end{itemize}

\textbf{Account-level default tier:}
\begin{itemize}
    \item Hot lub Cool (nie może być Archive)
    \item Wszystkie nowe blobs domyślnie w tym tier
    \item Można override per-blob
\end{itemize}
\end{infobox}

\section{Queue Storage}

\subsection{Podstawy}

\begin{infobox}{Queue Storage - Message Queue}
\textbf{Definicja:} Simple, reliable message queue service dla asynchronous communication

\textbf{Architektura:}
\begin{center}
Producer → Queue → Consumer
\end{center}

\textbf{Zastosowanie:}
\begin{itemize}
    \item Decoupling components
    \item Asynchronous processing
    \item Load leveling (buffering bursts)
    \item Work distribution
    \item Task scheduling
\end{itemize}

\textbf{Benefits:}
\begin{itemize}
    \item Resilience - jeśli consumer down, messages czekają
    \item Scalability - multiple consumers
    \item Flexibility - producer i consumer niezależne
\end{itemize}
\end{infobox}

\subsection{Charakterystyka}

\begin{infobox}{Queue Storage - Specyfikacja}
\textbf{Message properties:}
\begin{itemize}
    \item Max message size: 64 KB
    \item Base64 encoded (faktycznie ~48 KB użytecznego payloadu)
    \item Text format (zwykle JSON)
\end{itemize}
\end{infobox}

\end{document}